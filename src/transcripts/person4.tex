\section{Transkript - Person 4}
\sloppy
\texttt{\begin{itemize}[]
        \setlength\itemsep{0.02em}
        \linenumbers
        \item {-------------------------} [Intro] {-------------------------}
        \item {---------------------} [FBA - Tutorial] {---------------------}
        \item \interview{P4} Der Hase ist kein Gegenstand, sondern ein Lebewesen.
              Spaß beiseite, aber das ist soweit verständlich.
        \item \interview{P4} Ist mit der Anzahl nur der Gegenstand Apfel gemeint, oder auch die anderen?
        \item \interview{IV} Auch die anderen Gegenstände.
        \item \interview{P4} Dann hätte ich zunächst drei die pflanzlich sind, aber der Rindfleischburger gehört anscheinend auch dazu.
              Kannst du mir erklären, wieso das der Fall ist?
        \item \interview{IV} Wir hatten vorhin das Beispiel mit dem Schwein, welches sowohl Pflanzenfresser als auch Fleischfresser ist.
              Beim Rindfleischburger ist es ähnlich.
              Die Zusammensetzung des Brötchens ist pflanzlich, aber das Fleisch ist tierisch.
        \item \interview{P4} Alles klar. Dann hätte das vier und oben alle Nahrungsmittel wären sechs.
        \item \interview{P4} Hafer gehört zu Getreide, pflanzliche Nahrungsmittel und Nahrungsmittel.
        \item \interview{P4} Wenn ich das richtig verstanden habe, dann gehören all diese Artikel zu Staat und Industrie.
              Die Zahlen verstehe ich, glaube ich, aber das ist trotzdem etwas kompliziert.
        \item {--------------------------} [FBA] {--------------------------}
        \item \interview{P4} Das hovern mit dem Mauszeiger finde ich noch etwas verwirrend.
              Wo genau soll ich hovern?
        \item \interview{IV} Das ist nur in der Legende selbst möglich.
        \item \interview{P4} Achso. Das war mir nicht ganz klar.
        \item \interview{P4} Dann schau ich mich jetzt mal im Markt um.
              Gut, dann gucken wir mal, ob sich ein E-Auto wirklich lohnt.
              Also klicke ich mal auf den Artikel \flqq Stromverbrauch: So viel verbraucht ein E-Auto wirklich\frqq{}.
        \item \interview{P4} Hier sehe ich jetzt, dass der Artikel zu Staat und Industrie gehört.
        \item \interview{P4} Was ist denn hier passiert?
              Ich habe auf Staat geklickt und sehe 0 Artikel.
              Oh, moment ich glaube ich habe falsch gedrückt.
              Aber dann würde ich jetzt \flqq Rohstoffe für E-Auto Akku\frqq{} klicken.
        \item \interview{IV} Kannst du deine Entscheidung für diesen Artikel noch kurz begründen?
        \item \interview{P4} Ja, weil Akkus offensichtlich sehr wichtig sind für Elektroautos.
              Ich habe mal selbst zu diesem Thema recherchiert und gelesen wie umweltschädlich die Produktion von Akkus ist.
        \item \interview{P4} Dann würde ich auf Grün klicken. Damit ist Klima und Umwelt gemeint?
        \item \interview{IV} Genau. Also im Detail musst du die Werte hinter diesen Welten nicht verstehen, aber Grün steht hier für Umwelt.
        \item \interview{P4} Alles klar. Dann hätte ich das vielleicht auch so genannt.
              Grün könnte man sonst einfach mit der politischen Partei verwechseln.
        \item \interview{P4} Als nächsten Artikel würde ich jetzt \flqq Neue Elektro-Studie unter der Lupe\frqq{} klicken.
              Der Grund dafür ist, dass Studien oftmals gutes Wissen mit sich bringen und die Zukunft zeigen.
              Zum Beispiel, welche Schäden die Produktion von E-Autos mit sich bringen kann.
        \item \interview{P4} Dann klicke ich mal auf Markt.
              So ein E-Auto muss ja auch irgendwie bezahlt werden.
              Und wie ich weiß sind die nicht gerade billig.
              Also so mein aktueller Stand.
              Da ist natürlich ein Artikel wie \flqq E-Auto Förderung 2021\frqq{} interessant.
        \item \interview{P4} Dann würde ich \flqq autonom. Fahren Erfolg oder Flop?\frqq{} klicken.
              Das klingt eigentlich ganz interessant.
        \item \interview{IV} Kannst du das noch etwas näher beleuchten?
        \item \interview{P4} Einfach allgemein, um zu schauen, was die anderen zu diesem Thema denken.
              Hier steht jetzt aber traurigerweise, dass bei diesem Artikel alles positiv ist.
              Finde ich Schade, weil ich gerne auch negative Meinungen dazu in diesem Artikel gelesen hätte.
        \item \interview{IV} Du hast jetzt hauptsächlich nur mit den Kategorien gearbeitet. Gibt es dafür einen bestimmten Grund?
        \item \interview{P4} Das ist eine sehr gute Frage.
              Das weiß ich selber nicht so genau.
              Aber ich denke mal, weil mir einfach mehr angezeigt wird.
              Also einfach, weil die Übersicht besser ist.
        \item {---------------------} [Aufgaben - FBA] {---------------------}
        \item \interview{IV} Finde und klicke den Artikel \flqq Harter Schlag für Hersteller Plugin Prämie fällt weg\frqq{}.
        \item \interview{P4} Hier würde ich dann genau den gerade genannten Vorteil ausnutzen.
              Also ich kann einfach oben auf den Knoten klicken und sehe dann die ganze Übersicht.
        \item \interview{IV} Und dann jetzt noch mal den Artikel \flqq Deutlich sauberer als gedacht\frqq{}.
        \item \interview{P4} Hab ich.
        \item \interview{IV} Welcher Artikel ist am ähnlichsten zu \flqq Deutlich sauberer als gedacht?\frqq{}?
        \item \interview{P4} Also ich würde sagen \flqq Von wegen nur das Klima retten\frqq{}.
        \item \interview{IV} Welcher Artikel unterscheidet sich am meisten von \flqq Deutlich sauberer als gedacht?\frqq{}?
        \item \interview{P4} \flqq Rohstoffe für E-Auto Akkus\frqq{}, weil es da um die Akkus geht und das eigentlich negativ sein müsste.
        \item {----------------------} [QUESI - FBA] {----------------------}
        \item {-------------------} [AttrakDiff2 - FBA] {-------------------}
        \item {---------------------} [Listenansicht] {---------------------}
        \item {----------------} [Aufgaben - Listenansicht] {----------------}
        \item \interview{IV} Finde und klicke den Artikel \flqq Strom, Trassen, Verteilernetze\frqq{}
        \item \interview{IV} Finde und klicke den Artikel \flqq Autobauer als Software-Riesen\frqq{}
        \item \interview{IV} Welcher Artikel ist am ähnlichsten zu \flqq Autobauer als Software-Riesen\frqq{}
        \item \interview{IV} Welcher Artikel unterscheidet sich am meisten von \flqq Autobauer als Software-Riesen\frqq{}.
        \item {-----------------} [QUESI - Listenansicht] {-----------------}
        \item {--------------} [AttrakDiff2 - Listenansicht] {--------------}
        \item {-----------------------} [Fragebogen] {----------------------}
        \item \interview{IV} Welche Darstellungsformen für Online-News-Artikel sind Dir bekannt?
        \item \interview{IV} Wie häufig besuchst Du News-Webseiten?
        \item \interview{IV} Inwieweit hast du bereits Vorerfahrung zum Thema E-Mobilität?
        \item \interview{IV} Könntest Du Dir vorstellen einer der beiden gezeigten Darstellungsformen persönlich zu nutzen?
        \item \interview{IV} Hast Du bereits mit Graphen gearbeitet?
        \item \interview{IV} Denkst Du, dass die Dir gezeigte neue Darstellungsform ein diverses Bild der Elektromobilität zeigt?
        \item \interview{IV} Wie beeinflusst das Aussehen einer Webseite Deine Entscheidung, welche Informationen Du liest und wie lange Du auf der Seite bleibst?
        \item {------------------} [Demografische Daten] {------------------}
    \end{itemize}}
\nolinenumbers
