\section{Transkript - Person 3}
\sloppy
\texttt{\begin{itemize}[]
            \setlength\itemsep{0.02em}
            \resetlinenumber
            \linenumbers
            \item \taskseparator{Intro}
            \item \taskseparator{Listenansicht}
            \item \interview{P3} Hier sind einige interessante Artikel dabei.
                  \flqq Elektro-Zwang?\frqq{} und \flqq Umverteilung von Arm nach Grün\frqq{} würde mich beides aufregen, weswegen ich die Artikel wahrscheinlich klicken würde.
                  Aber ich fange mal mit \flqq Elektro-Zwang?\frqq{} an.
            \item \interview{IV} Könntest du dann auch am besten immer erläutern, weswegen du die Artikel klickst?
            \item \interview{P3} Ja, klar. Also \flqq Elektro-Zwang?\frqq{} klingt offensichtlich ziemlich negativ.
                  Ein Zwang suggiert für mich, dass ich betroffen werden könnte oder etwas machen muss, das ich nicht möchte.
                  Deshalb möchte ich wissen: Was ist das für ein Zwang? Worum geht es genau? Und werde ich betroffen?
            \item \interview{P3} Dieser Artikel klingt sehr dramatisch.
            \item \interview{IV} Welchen genau meinst du damit?
            \item \interview{P3} \flqq Willkommen in der grünen Flammenhölle\frqq{}.
                  Deswegen klicke ich den jetzt, weil der Artikel dramatisch klingt und mir die grüne Politik nicht zusagt.
            \item \interview{P3} Dann nehme ich jetzt als nächstes \flqq Umverteilung von Arm nach Grün\frqq{}, weil mich das Thema gerade interessiert.
                  Also allgemein die grüne Klimapolitik finde ich sehr problematisch und, dass dadurch viele Unkosten entstehen unter dem Deckmantel Klimaschutz.
                  Viele Preise schießen deswegen enorm in die Höhe und das betrifft natürlich auch die E-Mobilität.
                  Da interessieren mich diese Artikel natürlich, weil ich mich informieren möchte, ob das stimmt oder allgemein öffentlich Kritik an der grünen Politik geäußert wird.
                  Also ich habe im Vorfeld natürlich bereits ein bestimmtes Bild worum es gehen könnte bei den Titeln und wollte mir deswegen die Artikel anschauen.
            \item \interview{P3} \flqq Fürs Klima und gegen China\frqq{} finde ich interessant, weil beide große Themen sind.
                  China ist natürlich ein riesiger Akteur und die kontroverse Klimapolitik ist natürlich auch ein großes Thema.
                  Hier wäre ich daran interessiert, ob und was China falsch macht.
                  Möglicherweise könnte hier behandelt werden, dass China ein großer Umweltverschmutzer ist oder, dass es um Handelsbeziehungen geht.
            \item \interview{IV} Geht es bei diesen Handelsbeziehungen auch um die E-Mobilität?
                  Also nur, dass ich das richtig verstehe und wir sicherstellen, dass wir uns weiter auf das Thema E-Mobilität konzentrieren.
            \item \interview{P3} Ja, genau. Also das war auch damit gemeint.
            \item \interview{P3} Als nächsten Artikel würde ich \flqq Autobauer als Software-Riesen\frqq{} klicken, weil ich generell daran interessiert bin zu sehen, wie sich die Automobilindustrie entwickelt.
                  Ich finde das Thema sehr spannend, dass der Fokus viel stärker auf Software aktuell liegt.
                  Das klingt natürlich auch sehr interessant und ich würde mich gerne informieren wollen, welche Änderungen im Auto gemacht werden.
            \item \taskseparator{Aufgaben - Listenansicht}
            \item \interview{IV} Finde und klicke den Artikel \flqq Strom, Trassen, Verteilernetze\frqq{}.
            \item \interview{P3} Anklicken?
            \item \interview{IV} Ja genau, anklicken.
            \item \interview{P3} Alles klar.
            \item \interview{IV} Finde und klicke den Artikel \flqq Autobauer als Software-Riesen\frqq{}.
            \item \interview{P3} Okay.
            \item \interview{IV} Welcher Artikel ist am ähnlichsten zu \flqq Autobauer als Software-Riesen\frqq{}?
            \item \interview{P3} Okay, schwierig. Software-Riese. Ich würde sagen \flqq Herstellungskosten gecheckt\frqq{}
            \item \interview{IV} Welcher Artikel unterscheidet sich am meisten von \flqq Autobauer als Software-Riesen\frqq{}?
            \item \interview{P3} Also zuallererst fällt mir \flqq Diess kontert Reitzle\frqq{} ins Auge, weil das eventuell ein Schlagabtausch zwischen zwei bekannten Personen zu sein scheint.
                  Wenn ich jedoch mehr dazu wüsste, könnte es sein, dass ich das jetzt nicht so auswählen würde.
            \item \taskseparator{QUESI - Listenansicht}
            \item \interview{P3} Allgemein kann ich sagen, dass ich eigentlich mit dem System erreicht habe was ich wollte.
                  Das ist einfach und zugänglich und ich konnte mich schnell zurechtfinden.
                  Ich würde jetzt davon ausgehen, dass ich alles lesen konnte und ich genauso informieren konnte wie ich es wollte.
            \item \taskseparator{AttrakDiff2 - Listenansicht}
            \item \taskseparator{FBA - Tutorial}
            \item \interview{P3} Ich würde schätzen vier pflanzliche Nahrungsmittel.
                  Gegenstände bei Nahrungsmittel sind die vier von pflanzliche Nahrungsmittel plus zwei von tierische Nahrungsmittel plus Rindfleischburger.
                  Dann wären vier plus drei gleich sieben.
            \item \interview{P3} Und Hafer kann allen drei Kategorien angehören. Also Nahrungsmittel, pflanzliche Nahrungsmittel und Getreide.
            \item \interview{P3} Artikel können aus unterschiedlichen Perspektiven betrachtet werden. Ich finde, das ist ein super interessantes Thema.
            \item \interview{P3} Dieses plus eins rechnen finde ich aktuell noch etwas schwierig, aber ich glaube so grob habe ich es verstanden.
            \item \taskseparator{FBA}
            \item \interview{P3} Also im Prinzip kann ich anklicken, was mich persönlich interessiert, korrekt?
            \item \interview{IV} Ja genau. Das ist richtig.
            \item \interview{P3} Also hier würde mich \flqq Von wegen nur das Klima retten\frqq{} interessieren.
            \item \interview{P3} Hier sehe ich jetzt auch, welche Kategorien dieser Artikel angehört.
                  Die einzelnen Farben wurden mir ja bereits im Tutorial erklärt.
                  Was mir hier aber fehlt ist, dass die Sätze wieder im Text angezeigt werden.
            \item \interview{P3} So, ich möchte mich über E-Mobilität informieren.
                  Natürlich klicke ich dann auch hier wieder auf Grün.
                  \flqq Deutlich sauberer als gedacht\frqq{} klingt sehr positiv.
                  Wenn ich jetzt interessiert daran wäre ein Elektroauto zu kaufen, dann fühle ich mich natürlich gut, wenn es sogar noch sauberer ist als gedacht.
            \item \interview{P3} Dann klicke ich mal die Kombination aus Staat und Grün.
                  Der Artikel \flqq Harter Schlag für Hersteller Plug-in Prämie fällt weg\frqq{} klingt natürlich auch sehr interessant, weil die Prämie für Elektroautos wegfällt.
                  Wenn eine solche Prämie wegfällt, könnte es natürlich sein, dass Elektroautos wieder unattraktiver werden.
                  Deshalb würde ich jetzt gerne mehr darüber erfahren.
            \item \interview{P3} Und weil ich den Artikel vorhin schon entdeckt habe und mich auch dafür interessiere, würde ich jetzt noch \flqq E-Auto Förderung 2021\frqq{} anklicken.
                  Schließlich passt das thematisch gut zum vorherigem Artikel.
            \item \interview{P3} Und ich kann natürlich auch hier den Knoten mit der 13 klicken, um alle Artikel zu sehen.
                  Ah perfekt. Das klappt super.
            \item \interview{P3} Ich finde viele Artikel interessant, aber ich glaube \flqq Lohnt sich E-Auto bei den Strompreisen noch?\frqq{} ist der Artikel, der mich am meisten interessiert.
                  Ich meine gerade das Thema mit dem Strompreisen ist, ziemlich aktuell.
                  Verknüpft mit dem Thema Elektroauto auch sehr spannend.
            \item \taskseparator{Aufgaben - FBA}
            \item \interview{IV} Finde und klicke den Artikel \flqq Harter Schlag für Hersteller Plug-in Prämie fällt weg\frqq{}.
            \item \interview{P3} Ich würde den Artikel bei Staat und Markt irgendwo vermuten.
                  Aber ist in diesem Test gefordert, dass ich den Artikel direkt finden muss?
                  Weil dann würde ich einfach oben auf den Knoten mit der 13 klicken.
            \item \interview{IV} Es ist nicht wichtig, dass du den Artikel direkt findest.
                  Es wäre aber wichtig, wenn du einfach dein Vorgehen beschreibst wie du es gerade gemacht hast.
            \item \interview{P3} Alles klar. Dann würde ich das über diesen Knoten hier oben machen.
            \item \interview{IV} Finde und klicke den Artikel \flqq Deutlich sauberer als gedacht\frqq{}.
            \item \interview{P3} Ebenfalls einfach, wenn ich über diese Gesamtübersicht gehe.
                  Und ansonsten würde ich den Artikel bei Grün vermuten.
            \item \interview{IV} Welcher Artikel ist am ähnlichsten zu \flqq Deutlich sauberer als gedacht\frqq{}?
            \item \interview{P3} \flqq Rohstoffe für E-Auto Akku\frqq{} wäre auf alle Fälle zusammenhängend.
                  Aber ob das so korrekt ist weiß ich nicht.
            \item \interview{IV} Welcher Artikel unterscheidet sich am meisten von \flqq Deutlich sauberer als gedacht\frqq{}?
            \item \interview{P3} \flqq Revolution oder Wegwerfauto?\frqq{} könnte gerade mit dem zweiten Part suggerieren, dass es nicht deutlich sauberer ist als gedacht.
                  Deswegen wähle ich den Artikel als den, welcher sich am meisten unterscheidet.
            \item \taskseparator{QUESI - FBA}
            \item \taskseparator{AttrakDiff2 - FBA}
            \item \taskseparator{Fragebogen}
            \item \interview{IV} Welche Darstellungsformen für Online-News-Artikel sind dir bekannt?
            \item \interview{P3} Aus dem Kopf heraus würde ich sagen, dass eigentlich alles Listen sind.
                  Also auch, wenn ich Artikel auf Google News am Handy lese ist das ja eine große Liste.
            \item \interview{IV} Wie häufig besuchst du News-Webseiten?
            \item \interview{P3} Momentan eigentlich gar nicht. Also vielleicht einmal pro Woche.
            \item \interview{IV} Inwieweit hast du bereits Vorerfahrung zum Thema E-Mobilität?
            \item \interview{P3} Ich bin schonmal in einem Tesla Beifahrer gewesen.
                  Ein bisschen für das Thema habe ich mich auch persönlich interessiert und mir ein paar Artikel angeschaut.
                  Also Kontakt mit dem Thema hatte ich schon.
            \item \interview{IV} Könntest du dir vorstellen einer der beiden gezeigten Darstellungsformen persönlich zu nutzen?
            \item \interview{P3} Die Liste auf alle Fälle.
            \item \interview{IV} Könntest du auch erläutern wieso?
            \item \interview{P3} Ich finde die Liste wesentlich zugänglicher.
                  Es ist simpel und angenehm für die Augen, besonders, weil mein Kopf sich nicht anstrengen muss.
                  Wenn ich ein Thema finden möchte, möchte ich nicht noch darüber nachdenken wie ich dahin navigiere.
                  Ich will viel lieber eine ein paar Optionen präsentiert bekommen ohne mich großartig damit auseinandersetzen zu müssen.
            \item \interview{IV} Hast du bereits mit Graphen gearbeitet?
            \item \interview{P3} Ja.
            \item \interview{IV} Denkst du, dass die dir gezeigte neue Darstellungsform ein diverses Bild der Elektromobilität zeigt?
            \item \interview{P3} Tatsächlich schon. Die Liste zum Beispiel präsentiert Inhalte relativ simpel.
                  Du hast Optionen, welche du anklicken kannst, aber diese sind nicht kategorisiert.
                  Die Kategorisierung macht aber einen großen Unterschied.
                  Es hilft zu verstehen in welche Richtung es geht und dann auch wie Artikel und Kategorien zusammenhängen.
            \item \interview{IV} Wie beeinflusst das Aussehen einer Webseite Deine Entscheidung, welche Informationen du liest und wie lange du auf der Seite bleibst?
            \item \interview{P3} Ich würde sagen, dass es wichtig ist, dass die Webseite übersichtlich ist.
                  Es muss ebenfalls angenehm zum Lesen sein und das Farbschema muss stimmen.
            \item \interview{IV} Du hattest das vorhin schon so ähnlich erwähnt und da würde ich gerne nochmal nachhaken.
                  Du hast beschrieben, dass die Liste weniger anstrengend ist.
                  Kannst du das nochmal genauer erklären?
            \item \interview{P3} Das Design lenkt mich nicht ab.
                  Also die Liste ist sauber aufgelistet und es spiegelt halt diese Gemütlichkeit beim Navigieren wider.
                  Ich muss mir im Vorfeld nicht überlegen, in welche Richtung Artikel oder Kategorien gehen.
                  Ich kann also einfach auf einen Artikel klicken, welcher interessant klingt.
                  Das geht auch in die Richtung wie mit sozialen Medien.
                  Meistens will ich mich eher kurz informieren und nicht lange auf solchen Webseiten aufhalten.
                  Also es ist halt nicht so wie früher, wo Leute ihre Zeitung aufschlagen und dann eine Stunde lang lesen.
                  Vielleicht möchte ich auch nur einmal fünf Minuten so gucken, was es Neues gibt und da würde die Navigation in der Liste mir da entgegenkommen.
            \item \taskseparator{Demografische Daten}
            \item \interview{P3} 26, männlich, Abitur, Tischler
      \end{itemize}}
\nolinenumbers
