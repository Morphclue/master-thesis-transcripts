\section{Transkript - Person 3}
\sloppy
\texttt{\begin{itemize}[]
        \setlength\itemsep{0.02em}
        \linenumbers
        \item {-------------------------} [Intro] {-------------------------}
        \item {---------------------} [Listenansicht] {---------------------}
        \item \interview{P3} Hier sind einige interessante Artikel dabei.
        \flqq Elektro-Zwang?\frqq{} und \flqq Umverteilung von Arm nach Grün\frqq{} würde mich beides aufregen, weswegen ich die Artikel wahrscheinlich klicken würde.
        Aber ich fange mal mit \flqq Elektro-Zwang?\frqq{} an.
        \item \interview{IV} Könntest du dann auch am besten immer erläutern, weswegen Du die Artikel klickst?
        \item \interview{P3} Ja, klar. Also \flqq Elektro-Zwang?\frqq{} klingt offensichtlich ziemlich negativ.
        Ein Zwang suggiert für mich, dass ich betroffen werden könnte oder etwas machen muss, das ich nicht möchte.
        Deshalb möchte ich wissen: Was ist das für ein Zwang? Worum geht es genau? Und werde ich betroffen?
        \item \interview{P3} Dieser Artikel klingt sehr dramatisch.
        \item \interview{IV} Welchen genau meinst du damit?
        \item \interview{P3} \flqq Willkommen in der grünen Flammenhölle\frqq{}.
        Deswegen klicke ich den jetzt, weil der Artikel dramatisch klingt und mir die grüne Politik nicht zusagt.
        \item \interview{P3} Dann nehme ich jetzt als nächstes \flqq Umverteilung von Arm nach Grün\frqq{}, weil mich das Thema gerade interessiert.
        Also allgemein die grüne Klimapolitik finde ich sehr problematisch und, dass dadurch viele Unkosten entstehen unter dem Deckmantel Klimaschutz.
        Viele Preise schießen deswegen enorm in die Höhe und das betrifft natürlich auch die E-Mobilität.
        Da interessieren mich diese Artikel natürlich, weil ich mich informieren möchte, ob das stimmt oder allgemein öffentlich Kritik an der grünen Politik geäußert wird.
        Also ich habe im Vorfeld natürlich bereits ein bestimmtes Bild worum es gehen könnte bei den Titeln und wollte mir deswegen die Artikel anschauen.
        \item \interview{P3} \flqq Fürs Klima und gegen China\frqq{} finde ich interessant, weil beide große Themen sind.
        China ist natürlich ein riesiger Akteur und die kontroverse Klimapolitik ist natürlich auch ein großes Thema.
        Hier wäre ich daran interessiert, ob und was China falsch macht.
        Möglicherweise könnte hier behandelt werden, dass China ein großer Umweltverschmutzer ist oder, dass es um Handelsbeziehungen geht.
        \item \interview{IV} Geht es bei diesen Handelsbeziehungen auch um die E-Mobilität? 
        Also nur, dass ich das richtig verstehe und wir sicherstellen, dass wir uns weiter auf das Thema E-Mobilität konzentrieren. 
        \item \interview{P3} Ja, genau. Also das war auch damit gemeint.
        \item \interview{P3} Als nächsten Artikel würde ich \flqq Autobauer als Software-Riesen\frqq{} klicken, weil ich generell daran interessiert bin zu sehen, wie sich die Automobilindustrie entwickelt.
        Ich finde das Thema sehr spannend, dass der Fokus viel stärker auf Software aktuell liegt.
        Das klingt natürlich auch sehr interessant und ich würde mich gerne informieren wollen, welche Änderungen im Auto gemacht werden.
        \item {----------------} [Aufgaben - Listenansicht] {----------------}
        \item \interview{IV} Finde und klicke den Artikel \flqq Strom, Trassen, Verteilernetze\frqq{}
        \item \interview{P3} Anklicken?
        \item \interview{IV} Ja genau, anklicken.
        \item \interview{P3} Alles klar.
        \item \interview{IV} Finde und klicke den Artikel \flqq Autobauer als Software-Riesen\frqq{}
        \item \interview{P3} Okay.
        \item \interview{IV} Welcher Artikel ist am ähnlichsten zu \flqq Autobauer als Software-Riesen\frqq{}
        \item \interview{P3} Okay, schwierig. Software-Riese. Ich würde sagen \flqq Herstellungskosten gecheckt\frqq{}
        \item \interview{IV} Welcher Artikel unterscheidet sich am meisten von \flqq Autobauer als Software-Riesen\frqq{}.
        \item \interview{P3} Also zuallererst fällt mir Diess kontert Reitzle ins Auge, weil das eventuell ein Schlagabtausch zwischen zwei bekannten Personen zu sein scheint.
        Wenn ich jedoch mehr dazu wüsste, könnte es sein, dass ich das jetzt nicht so auswählen würde.
        \item {-----------------} [QUESI - Listenansicht] {-----------------}
        \item {--------------} [AttrakDiff2 - Listenansicht] {--------------}
        \item {---------------------} [FBA - Tutorial] {---------------------}
        \item {--------------------------} [FBA] {--------------------------}
        \item {---------------------} [Aufgaben - FBA] {---------------------}
        \item \interview{IV} Finde und klicke den Artikel \flqq Harter Schlag für Hersteller Plugin Prämie fällt weg\frqq{}.
        \item \interview{IV} Und dann jetzt noch mal den Artikel \flqq Deutlich sauberer als gedacht\frqq{}.
        \item \interview{IV} Welcher Artikel ist am ähnlichsten zu \flqq Deutlich sauberer als gedacht?\frqq{}?
        \item \interview{IV} Welcher Artikel unterscheidet sich am meisten von \flqq Deutlich sauberer als gedacht?\frqq{}?
        \item {----------------------} [QUESI - FBA] {----------------------}
        \item {-------------------} [AttrakDiff2 - FBA] {-------------------}
        \item \interview{IV} Welche Darstellungsformen für Online-News-Artikel sind Dir bekannt?
        \item \interview{IV} Wie häufig besuchst Du News-Webseiten?
        \item \interview{IV} Inwieweit hast du bereits Vorerfahrung zum Thema E-Mobilität?
        \item \interview{IV} Könntest Du Dir vorstellen einer der beiden gezeigten Darstellungsformen persönlich zu nutzen?
        \item \interview{IV} Hast Du bereits mit Graphen gearbeitet?
        \item \interview{IV} Denkst Du, dass die Dir gezeigte neue Darstellungsform ein diverses Bild der Elektromobilität zeigt?
        \item \interview{IV} Wie beeinflusst das Aussehen einer Webseite Deine Entscheidung, welche Informationen Du liest und wie lange Du auf der Seite bleibst?
        \item {------------------} [Demografische Daten] {------------------}
    \end{itemize}}
\nolinenumbers
