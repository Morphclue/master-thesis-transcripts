\section{Transkript - Person 6}
\sloppy
\texttt{\begin{itemize}[]
        \setlength\itemsep{0.02em}
        \linenumbers
        \item {-------------------------} [Intro] {-------------------------}
        \item {---------------------} [FBA - Tutorial] {---------------------}
        \item \interview{P6} Was genau ist ein Knoten?
        \item \interview{IV} Diese Kugel hier wird Knoten genannt.
        \item \interview{P6} Ich verstehe die Aufgabe mit den Gegenständen nicht ganz.
              Ich muss quasi sagen wie viel Gegenstände hier bei pflanzliche Nahrungsmittel sind, oder?
        \item \interview{IV} Ja, genau.
        \item \interview{P6} Dann sehe ich drei. Oder vielleicht auch vier?
              Also wieso gehört der Rindfleischburger zu pflanzlichen Nahrungsmitteln?
        \item \interview{IV} Der Rindfleischburger besteht aus mehreren Bestandteilen. Das Brötchen ist pflanzlich, das Fleisch ist tierisch.
        \item \interview{P6} Okay. Das war, glaube ich, vorhin auch schon so.
        \item \interview{IV} Ja, genau. Beim Beispiel mit den Tieren war es mit dem Schwein genauso.
        \item \interview{P6} Gut.
              Dann habe ich das verstanden.
              Aber es ist trotzdem schwierig das direkt alles im Kopf zu behalten.
        \item \interview{P6} Bei Nahrungsmitteln haben wir vier Knoten die darunterliegen. Also vier.
        \item \interview{IV} Und wenn du jetzt nur die Gegenstände beachtest?
        \item \interview{P6} Dann hätten wir fünf, klar.
        \item \interview{P6} Hafer gehört zu pflanzlichen Nahrungsmitteln und Getreide. Achja, und zu Nahrungsmittel.
        \item \interview{P6} Diese zwei Ansichten finde ich ziemlich kompliziert.
        \item \interview{IV} Was verstehst du denn genau unter diesen Ansichten?
        \item \interview{P6} Die erste Ansicht zeigt die Knoten und ein paar Artikel und die zweite Ansicht zeigt dann vermutlich alle Artikel, die darunter liegen.
              Aber wie die Zahlen sich zusammensetzen finde ich noch kompliziert.
              Vielleicht ergibt sich das ja noch.
        \item {--------------------------} [FBA] {--------------------------}
        \item \interview{P6} Wo genau sind die Artikel?
        \item \interview{IV} Die Artikel sind wie vorhin im Tutorial in den Knoten versteckt.
              Du kannst hier nichts wirklich kaputt machen, also klicke dich einfach etwas durch.
        \item \interview{P6} Ich habe hier jetzt bei Staat auf negativ geklickt.
              Dann schauen wir uns mal an, was passiert.
              Ich klicke jetzt auf \flqq Harter Schlag für Hersteller Plug-in Prämie fällt weg\frqq{}.
              Aber hier steht, dass es positiv für den Staat ist.
        \item \interview{IV} Genau.
              Die Klicks auf die unterschiedlichen Farben in der Kategorie geben nur das Verhältnis für den Knoten selbst an.
        \item \interview{P6} Das finde ich etwas verwirrend.
              Also ich verstehe, dass man so etwas erfährt, ob eine Kategorie positiv oder negativ ist.
              Aber das hilft mir nicht wirklich weiter.
        \item \interview{IV} Alles klar.
              Danke für deinen Input.
              Kannst du mir dann vielleicht etwas mehr darüber erzählen, weswegen du dich für diesen Artikel entschieden hast?
        \item \interview{P6} Ich wollte mir etwas aus der Perspektive Staat ansehen. Aber eigentlich eher negativ.
        \item \interview{IV} Hättest du Interesse daran, wenn du auf Staat klickst, dass Artikel farblich markiert werden, welche sich dann positiv, negativ oder beides auf den Staat beziehen?
        \item \interview{P6} Das klingt nach einer guten Idee.
              Ich weiß noch nicht, ob das dann auch wirklich hilft, aber ich stelle mir das nützlicher vor als wie es gerade ist.
        \item \interview{P6} Als Nächstes lese ich mir \flqq Von wegen nur das Klima retten\frqq{} durch.
              Wenn ich jetzt nämlich schon in dieser Staat-Kategorie bin, dann würde ich da jetzt auch mehr zu lesen wollen.
        \item \interview{P6} Der Artikel \flqq autonom. Fahren Erfolg oder Flop?\frqq{} klingt interessant.
              Den würde ich einfach mal anklicken.
        \item \interview{P6} \flqq Rohstoffe für E-Auto Akku\frqq{} wäre der nächste, weil ich interessiert daran bin wie die Akkus hergestellt werden.
        \item \interview{P6} Dann würde ich jetzt \flqq Auch ohne Berlin-Werk: Schon heute verdient dt. Autoindustrie\frqq{} lesen, weil das mit der Industrie zusammenhängt.
        \item {---------------------} [Aufgaben - FBA] {---------------------}
        \item \interview{IV} Finde und klicke den Artikel \flqq Harter Schlag für Hersteller Plugin Prämie fällt weg\frqq{}.
        \item \interview{P6} Den hatte ich vorhin schon. Also hier wäre er.
        \item \interview{IV} Und dann jetzt noch mal den Artikel \flqq Deutlich sauberer als gedacht\frqq{}.
        \item \interview{P6} \flqq Deutlich sauberer als gedacht\frqq{}?
              Den hatte ich noch nicht.
              Aber ich würde den jetzt bei Grün vermuten.
        \item \interview{IV} Welcher Artikel ist am ähnlichsten zu \flqq Deutlich sauberer als gedacht?\frqq{}?
        \item \interview{P6} Ich würde vermuten \flqq Von wegen nur das Klima retten\frqq{}.
        \item \interview{IV} Welcher Artikel unterscheidet sich am meisten von \flqq Deutlich sauberer als gedacht?\frqq{}?
        \item \interview{P6} \flqq E-Auto Förderung\frqq{}.
        \item {----------------------} [QUESI - FBA] {----------------------}
        \item {-------------------} [AttrakDiff2 - FBA] {-------------------}
        \item {---------------------} [Listenansicht] {---------------------}
        \item \interview{IV} Gibt es einen Grund, wieso du dich für \flqq Elektro-Zwang?\frqq{} entschieden hast?
        \item \interview{P6} Ein Zwang ist nicht immer eine gute Aussicht.
              Ich würde mich jetzt gerne informieren welchen Zwang das bedeutet und wer dahinter steckt.
        \item \interview{P6} \flqq Diess kontert Reitzle\frqq{}. Ich denke, dass dieser Artikel vielleicht etwas mit dem vorherigen zu tun hat.
              Also mein Vorgehen wäre, dass ich das Thema weiter verfolge.
        \item \interview{P6} \flqq Umverteilung von Arm nach Grün\frqq{} interessiert mich, weil es auch politisch motiviert zu sein scheint.
        \item \interview{P6} Dann wieder das gleiche Thema wie vorhin.
              Also der Artikel \flqq Willkommen in der grünen Flammenhölle\frqq{} beleuchtet ja ein ähnliches Thema.
              Ein Artikel reicht meistens nicht aus, weswegen ich da gerne auch noch einen zweiten Artikel dazu lese.
        \item \interview{P6} Jetzt vielleicht \flqq Autobauer als Software-Riesen\frqq{}.
              Der Titel klingt interessant.
        \item {----------------} [Aufgaben - Listenansicht] {----------------}
        \item \interview{IV} Finde und klicke den Artikel \flqq Strom, Trassen, Verteilernetze\frqq{}.
        \item \interview{P6} Ja.
        \item \interview{IV} Finde und klicke den Artikel \flqq Autobauer als Software-Riesen\frqq{}.
        \item \interview{P6} Ja.
        \item \interview{IV} Welcher Artikel ist am ähnlichsten zu \flqq Autobauer als Software-Riesen\frqq{}?
        \item \interview{P6} Ich glaube \flqq Strom, Trassen, Verteilernetze\frqq{}.
        \item \interview{IV} Welcher Artikel unterscheidet sich am meisten von \flqq Autobauer als Software-Riesen\frqq{}?
        \item \interview{P6} \flqq Willkommen in der grünen Flammenhölle\frqq{}.
        \item {-----------------} [QUESI - Listenansicht] {-----------------}
        \item {--------------} [AttrakDiff2 - Listenansicht] {--------------}
        \item {-----------------------} [Fragebogen] {----------------------}
        \item \interview{IV} Welche Darstellungsformen für Online-News-Artikel sind Dir bekannt?
        \item \interview{P6} Mit kleinen Ausschnitten, Videos, Fotos und Weitere in der Art.
        \item \interview{IV} Aber im Endeffekt als Liste dargestellt oder noch auf eine andere Art und Weise?
        \item \interview{P6} Als Liste.
        \item \interview{IV} Wie häufig besuchst Du News-Webseiten?
        \item \interview{P6} Zweimal täglich. Morgens und Mittags.
        \item \interview{IV} Inwieweit hast du bereits Vorerfahrung zum Thema E-Mobilität?
        \item \interview{P6} Ein wenig. Ich habe da nur ein paar Artikel gelesen.
        \item \interview{IV} Könntest Du Dir vorstellen einer der beiden gezeigten Darstellungsformen persönlich zu nutzen?
        \item \interview{P6} Vermutlich die erste Darstellungsform.
        \item \interview{IV} Also den Graphen?
        \item \interview{P6} Ja.
        \item \interview{IV} Hat das einen bestimmten Grund?
        \item \interview{P6} Einfach, weil das etwas Neues ist.
              Die Ansicht ist einfach viel interessanter.
              Also auch die Kategorisierung der Artikel finde ich sehr gut.
              Wenn man das öfters nutzen würde könnte man sich auch schnell einen Überblick verschaffen.
              Vielleicht wäre das Graph noch besser, wenn man den Filtern könnte.
        \item \interview{IV} Hast Du bereits mit Graphen gearbeitet?
        \item \interview{P6} Nicht wirklich gearbeitet, aber ich habe solche Graphen mal auf Webseiten gesehen.
        \item \interview{IV} Denkst Du, dass die Dir gezeigte neue Darstellungsform ein diverses Bild der Elektromobilität zeigt?
        \item \interview{P6} Ja.
        \item \interview{IV} Und weswegen?
        \item \interview{P6} Diese neuartige Form der Kategorisierung zeigt die Verteilung der Artikel an.
              Das ist ziemlich praktisch und zeigt übersichtlich die verschiedenen Themen an.
        \item \interview{IV} Wie beeinflusst das Aussehen einer Webseite Deine Entscheidung, welche Informationen Du liest und wie lange Du auf der Seite bleibst?
        \item \interview{P6} Das spielt definitiv eine Rolle, aber ich denke, dass die Inhalte der Artikel wichtiger sind.
        \item {------------------} [Demografische Daten] {------------------}
        \item \interview{P6} 50, männlich, Abitur, Maschinenführer in Print- und Druckmedien
    \end{itemize}}
\nolinenumbers
