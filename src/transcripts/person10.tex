\section{Transkript - Person 10}
\sloppy
\texttt{\begin{itemize}[]
        \setlength\itemsep{0.02em}
        \resetlinenumber
        \linenumbers
        \item \taskseparator{Intro}
        \item \taskseparator{FBA - Tutorial}
        \item \interview{P10} Es gibt vier pflanzliche Nahrungsmittel.
        \item \interview{P10} Und sieben Nahrungsmittel
        \item \interview{IV} Könntest du vielleicht noch erläutern, wie du auf die sieben Nahrungsmittel gekommen bist?
        \item \interview{P10} Ja, klar.
              Ich habe einfach die vier pflanzlichen Nahrungsmittel plus die zwei tierischen Nahrungsmittel und plus den Rindfleischburger genommen.
              Das wären dann sieben Nahrungsmittel.
        \item \interview{P10} Hafer ist ein Getreide, aber auch pflanzliches Nahrungsmittel und Nahrungsmittel.
        \item \interview{IV} Du bist jetzt durch mit dem Tutorial. War alles verständlich?
        \item \interview{P10} Ich denke schon.
        \item \taskseparator{FBA}
        \item \interview{P10} Ich klicke dann erstmal ein wenig random rum.
              Den Artikel \flqq Stromverbrauch: So viel verbraucht ein E-Auto wirklich\frqq{} finde ich ganz nett.
              Ich finde es auch ganz gut, dass beim Artikel lesen keine Werbung kommt, sondern direkt der Artikel kommt.
        \item \interview{P10} \flqq E-Auto Förderung\frqq{} wäre jetzt ein weiterer Artikel, über den ich mich schlaumachen würde.
              Aus den Gründen, dass ich vom Staat finanzielle Unterstützung bekomme, wenn ich mir ein E-Auto kaufe.
              Und billiger ist immer ganz nett.
              Aber ob E-Autos auch wirklich so Grün und positiv zu bewerten sind, weiß ich nicht.
              Das sehen wir ja noch in der Zukunft.
        \item \interview{IV} Wie würdest du die Kategorisierung denn allgemein bewerten?
              Also abgesehen davon jetzt, ob eine positive grüne Kategorisierung hier auch tatsächlich passt.
        \item \interview{P10} Finde ich wichtig.
              Das kann ja dabei helfen Artikel einfacher zu finden, welche einem persönlich gefallen.
              Ich kann mir aber auch vorstellen, dass es viel einfacher sein wird für die, ich nenne es mal, Hater bestimmte Kategorien anzugreifen.
        \item \interview{P10} Auf \flqq Neue E-Auto Förderung Tesla\frqq{} würde ich jetzt klicken, weil ich wieder eine Förderung erhalten würde.
              Gerade der Name Tesla ist ja auch ein sehr bekannter Markenname in dem Bereich.
        \item \interview{P10} \flqq Rohstoffe für E-Auto Akku\frqq{} klingt auch nicht verkehrt.
              Die Herstellungsbedingungen sind jetzt nicht allzu optimal, aber trotzdem denke ich, dass hier viel Interessantes stehen wird.
              Vermutlich dann auch etwas Kritisches zu den Herstellungsbedingungen.
        \item \interview{P10} Der Artikel \flqq Deutlich sauberer als gedacht\frqq{} klingt ziemlich nach Clickbait.
              Aber so wie ich mich kenne, würde ich trotzdem darauf klicken.
              Also vermutlich wird der Artikel nichts aussagen, aber ich würde einfach mal reinschauen.
        \item \taskseparator{Aufgaben - FBA}
        \item \interview{IV} Finde und klicke den Artikel \flqq Harter Schlag für Hersteller Plug-in Prämie fällt weg\frqq{}.
        \item \interview{P10} Vermutlich etwas geschummelt, aber ich habe den Artikel gefunden.
        \item \interview{IV} Könntest du vielleicht noch erläutern, weswegen du den obersten Knoten geklickt hast?
        \item \interview{P10} Naja. In dem Knoten sind halt alle Artikel.
        \item \interview{IV} Finde und klicke den Artikel \flqq Deutlich sauberer als gedacht\frqq{}.
        \item \interview{P10} Da würde ich wieder den hier nehmen.
        \item \interview{IV} Welcher Artikel ist am ähnlichsten zu \flqq Deutlich sauberer als gedacht\frqq{}?
        \item \interview{P10} Lass mich den mal kurz hier suchen.
              Okay, hab ihn.
              Ich würde sagen \flqq Von wegen nur das Klima retten\frqq{}.
              Das klingt ebenfalls nach einer grünen Thematik.
        \item \interview{IV} Welcher Artikel unterscheidet sich am meisten von \flqq Deutlich sauberer als gedacht\frqq{}?
        \item \interview{P10} Ich gehe hier mal auf Staat und \flqq Diesel soll 20 Cent teurer werden\frqq{} klingt doch gut.
        \item \interview{IV} Kannst du auch hier deine Wahl begründen?
        \item \interview{P10} Klar. Also ich würde einfach sagen, dass Diesel für mich persönlich nicht ganz nach Klima retten klingt.
              Es geht auch hier irgendwie mehr ums Geld, als ums Klima.
        \item \taskseparator{QUESI - FBA}
        \item \taskseparator{AttrakDiff2 - FBA}
        \item \taskseparator{Listenansicht}
        \item \interview{P10} Also das allererste, was mir ins Auge sticht, ist der Artikel \flqq 30.000 Euro geschenkt für einen Tesla\frqq{}.
              Da leuchten bei mir gleich die Dollarzeichen in den Augen.
              Ich würde ganz einfach auf den Artikel klicken, weil ich vielleicht die Möglichkeit habe diese 30.000 Euro zu bekommen.
        \item \interview{P10} \flqq Herstellungskosten gecheckt\frqq{} wäre das nächste.
        \item \interview{IV} Kannst du das begründen?
        \item \interview{P10} Ja. Einfach wieder aus dem Grund, weil ich die Herstellung interessant finde und ich gerne vor einem Kauf wissen möchte, wie teuer das alles ist.
              Vielleicht werde ich ja auch total abgezockt, wenn die Herstellung extrem günstig ist.
        \item \interview{P10} \flqq Willkommen in der grünen Flammenhölle\frqq{} klingt ehrlich gesagt auch wieder nach Clickbait.
              Ich kann mir darunter auch gar nichts vorstellen.
              Flammenhölle klingt aber ziemlich dramatisch und da bin ich gespannt was mich im Artikel erwartet.
        \item \interview{P10} \flqq Stromrationierung\frqq{} wäre für mich als nächstes wichtig, weil E-Autos ja offensichtlich Strom brauchen.
        \item \interview{P10} \flqq Strom, Trassen, Verteilernetze\frqq{} wäre genau aus dem gleichen Grund relevant.
        \item \taskseparator{Aufgaben - Listenansicht}
        \item \interview{IV} Finde und klicke den Artikel \flqq Strom, Trassen, Verteilernetze\frqq{}.
        \item \interview{P10} Okay, hab ich.
        \item \interview{IV} Finde und klicke den Artikel \flqq Autobauer als Software-Riesen\frqq{}.
        \item \interview{P10} Okay.
        \item \interview{IV} Welcher Artikel ist am ähnlichsten zu \flqq Autobauer als Software-Riesen\frqq{}?
        \item \interview{P10} Da würde ich \flqq Elektro-Zwang?\frqq{} nehmen.
        \item \interview{IV} Hat das einen bestimmten Grund?
        \item \interview{P10} Die bisherigen Hersteller von Elektroautos sind ja auch meistens Software-Riesen.
              Und ich denke, dass die durch die Software vermutlich schon viel besser sind als normale Autos.
              Also könnte man vielleicht schon von einem Zwang sprechen.
        \item \interview{IV} Welcher Artikel unterscheidet sich am meisten von \flqq Autobauer als Software-Riesen\frqq{}?
        \item \interview{P10} Das ist schwierig.
              Ich würde sagen \flqq Strom, Trassen, Verteilernetze\frqq{}.
              Die Autos benötigen zwar Strom, aber hier in dem Artikel geht es vermutlich eher um die Infrastruktur von Strom.
              Das ist ja etwas allgemeiner.
        \item \taskseparator{QUESI - Listenansicht}
        \item \taskseparator{AttrakDiff2 - Listenansicht}
        \item \taskseparator{Fragebogen}
        \item \interview{IV} Welche Darstellungsformen für Online-News-Artikel sind dir bekannt?
        \item \interview{P10} Ich kenne keine weiteren.
        \item \interview{IV} Wie häufig besuchst du News-Webseiten?
        \item \interview{P10} Einmal pro Tag.
        \item \interview{IV} Inwieweit hast du bereits Vorerfahrung zum Thema E-Mobilität?
        \item \interview{P10} Ein bisschen.
              Mal etwas gelesen oder dazu geschaut, aber auch nicht viel mehr.
        \item \interview{IV} Könntest du dir vorstellen einer der beiden gezeigten Darstellungsformen persönlich zu nutzen?
        \item \interview{P10} Beide.
              Also für diese neue Ansicht - Obwohl, fangen wir erstmal mit der Liste an.
              Die Liste ist kurz, knackig und einfach übersichtlich.
              Sieht schön aus und jeder versteht es.
              Mein persönlicher Favorit ist aber das mit den Kugeln.
              Das kann man eigentlich noch viel schöner gestalten.
              Und nehmen wir mal an, dass man das machen würde.
              Die Website würde mich einfach viel mehr dazu einladen länger zu bleiben und mich wirklich mit den Artikeln zu beschäftigen.
        \item \interview{IV} Hast du bereits mit Graphen gearbeitet?
        \item \interview{P10} Nein.
        \item \interview{IV} Denkst du, dass die dir gezeigte neue Darstellungsform ein diverses Bild der Elektromobilität zeigt?
        \item \interview{P10} Ja.
        \item \interview{IV} Kannst du das näher ausführen?
        \item \interview{P10} Bei der aktuellen Ansicht sieht man, dass viele verschiedene Kategorien irgendwie miteinander konkurrieren.
              Also das kommt viel mehr zum Vorschein.
        \item \interview{IV} Wie beeinflusst das Aussehen einer Webseite Deine Entscheidung, welche Informationen du liest und wie lange du auf der Seite bleibst?
        \item \interview{P10} Schon sehr stark. Also ich denke, wenn der Inhalt nicht spannend ist, dann bleibe ich auch nicht lange.
              Das Design der Website ist für mich aber genauso wichtig.
              Ich hatte ja erwähnt, dass mich Werbung nervt.
              Deswegen nutze ich auch einen Adblocker, weil das sonst für mich unerträglich wäre.
        \item \taskseparator{Demografische Daten}
        \item \interview{P10} 22, männlich, Fachhochschulreife, Mitarbeiter im Service-Team
    \end{itemize}}
\nolinenumbers
