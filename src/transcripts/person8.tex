\section{Transkript - Person 8}
\sloppy
\texttt{\begin{itemize}[]
            \setlength\itemsep{0.02em}
            \resetlinenumber
            \linenumbers
            \item {-------------------------} [Intro] {-------------------------}
            \item {---------------------} [FBA - Tutorial] {---------------------}
            \item \interview{P8} Pflanzliche Nahrungsmittel wären Apfel und Getreide.
            \item \interview{IV} Also Weizen und Hafer als Gegenstände. Wie sieht es mit dem Rindfleischburger aus?
            \item \interview{P8} Das weiß ich nicht.
            \item \interview{IV} Der Rindfleischburger ist pflanzlich, weil das Brötchen pflanzlich ist, aber auch tierisch, weil das Fleisch ein tierisches Nahrungsmittel ist.
            \item \interview{P8} Achso. Dann zählt der auch teils zu pflanzliche Nahrungsmittel.
            \item \interview{P8} Nahrungsmittel haben wir fünf.
            \item \interview{P8} Hafer gehört zu pflanzlichen Nahrungsmittel und Nahrungsmittel. Zu Getreide auch.
            \item \interview{IV} Alles klar, wir wären mit dem Tutorial fertig. Ist es für dich unverständlich?
            \item \interview{P8} Nein, ich glaube, alles ist gut.
                  Aber für mich ist das in dem Alter trotzdem etwas schwer.
            \item {--------------------------} [FBA] {--------------------------}
            \item \interview{P8} Dann klicke ich zuerst auf Staat.
                  \flqq Neue E-Auto Förderung Tesla\frqq{} klingt gut.
                  Das ist anscheinend in Markt, Industrie, Staat und Grün drinnen.
                  Finde ich gut.
            \item \interview{IV} Meinst du die Kategorisierungen? Also findest du die Kategorisierung an sich eine gute Idee?
            \item \interview{P8} Ja.
            \item \interview{P8} Hier bei Industrie würde ich auf das Dritte klicken.
                  Also \flqq Lohnt sich E-Auto bei den Strompreisen noch\frqq{}.
            \item \interview{IV} Weswegen hast du dich für diesen Artikel entschieden?
            \item \interview{P8} Wegen der Strompreise. Die steigen aktuell ja immer mehr.
            \item \interview{P8} Dann schaue ich jetzt mal auf Markt.
                  Hier würde ich \flqq Von wegen nur das Klima retten\frqq{} wählen.
            \item \interview{IV} Warum würdest du diesen Artikel wählen?
            \item \interview{P8} Klima ist aktuell auch ein wichtiges Thema. Da fände ich es nicht verkehrt mir diesen Artikel auch mal anzusehen.
            \item \interview{P8} Jetzt schaue ich mal auf Grün.
                  \flqq Deutlich sauberer als gedacht\frqq{} würde ich auswählen, weil das zum vorherigen Artikel passt.
                  Ein bisschen in Richtung Klima und Umwelt.
            \item \interview{P8} Dann würde ich den hier ganz unten klicken. \flqq E-Auto Förderung 2021\frqq{}, weil ich da vielleicht Geld erhalten könnte.
            \item {---------------------} [Aufgaben - FBA] {---------------------}
            \item \interview{IV} Finde und klicke den Artikel \flqq Harter Schlag für Hersteller Plug-in Prämie fällt weg\frqq{}.
            \item \interview{P8} Würde ich in Grün vermuten. Super, den Artikel gibt es hier.
            \item \interview{IV} Und dann jetzt noch mal den Artikel \flqq Deutlich sauberer als gedacht\frqq{}.
            \item \interview{P8} Auch in Grün. Hier vorne.
            \item \interview{IV} Welcher Artikel ist am ähnlichsten zu \flqq Deutlich sauberer als gedacht\frqq{}?
            \item \interview{P8} \flqq Revolution oder Wegwerfauto?\frqq{}.
            \item \interview{IV} Welcher Artikel unterscheidet sich am meisten von \flqq Deutlich sauberer als gedacht\frqq{}?
            \item \interview{P8} Vielleicht bei Markt etwas?
                  \flqq Lohnt sich E-Auto bei den Strompreisen noch?\frqq{} nehme ich.
            \item {----------------------} [QUESI - FBA] {----------------------}
            \item {-------------------} [AttrakDiff2 - FBA] {-------------------}
            \item {---------------------} [Listenansicht] {---------------------}
            \item \interview{P8} Der Artikel mit Tesla würde mich interessieren. \flqq 30.000 Euro geschenkt für einen Tesla\frqq{}.
            \item \interview{P8} Das sind jetzt wieder dieselben Artikel?
            \item \interview{IV} Genau, das sind die gleichen Artikel, die du vor dem ersten Klick in der Liste gesehen hast.
            \item \interview{P8} Dann \flqq Umverteilung von Arm nach Grün\frqq{}.
            \item \interview{IV} Weswegen hast du dich für diesen Artikel entschieden?
            \item \interview{P8} Ich wollte einfach mehr darüber herausfinden.
                  Also einfach mehr Informationen darüber.
                  Und vorhin mit den 30.000 Euro geschenkt, wollte ich auch mehr darüber erfahren.
            \item \interview{IV} Kannst du das vielleicht noch näher begründen?
                  Also, was genau wolltest du in diesem Artikel erfahren?
            \item \interview{P8} Naja, ich wollte wissen wie man an die 30.000 Euro kommt, wenn man sich einen Tesla kauft.
                  Wenn man so etwas liest, ist man natürlich neugierig.
            \item \interview{P8} \flqq Herstellungskosten gecheckt\frqq{}, weil ich wissen möchte, wie man das herstellt und wie viel das kostet.
            \item \interview{P8} Jetzt \flqq Stromrationierung\frqq{}. Hier würde ich klicken, weil es wieder um Strom geht und, weil es aktuell ist.
            \item \interview{P8} \flqq Fürs Klima und gegen China\frqq{} würde ich auch klicken, weil ich mich frage, wieso man gegen China ist.
            \item {----------------} [Aufgaben - Listenansicht] {----------------}
            \item \interview{IV} Finde und klicke den Artikel \flqq Strom, Trassen, Verteilernetze\frqq{}.
            \item \interview{P8} Hier auf der rechten Seite.
            \item \interview{IV} Finde und klicke den Artikel \flqq Autobauer als Software-Riesen\frqq{}.
            \item \interview{P8} Das ist auch hier.
            \item \interview{IV} Welcher Artikel ist am ähnlichsten zu \flqq Autobauer als Software-Riesen\frqq{}?
            \item \interview{P8} \flqq Herstellungskosten gecheckt\frqq{}.
            \item \interview{IV} Welcher Artikel unterscheidet sich am meisten von \flqq Autobauer als Software-Riesen\frqq{}?
            \item \interview{P8} \flqq Fürs Klima und gegen China\frqq{}.
            \item {-----------------} [QUESI - Listenansicht] {-----------------}
            \item {--------------} [AttrakDiff2 - Listenansicht] {--------------}
            \item {-----------------------} [Fragebogen] {----------------------}
            \item \interview{IV} Welche Darstellungsformen für Online-News-Artikel sind dir bekannt?
            \item \interview{P8} Ich kenne keine so wirklich.
            \item \interview{IV} Wie häufig besuchst du News-Webseiten?
            \item \interview{P8} Pro Woche vielleicht einmal.
            \item \interview{IV} Inwieweit hast du bereits Vorerfahrung zum Thema E-Mobilität?
            \item \interview{P8} Ich habe eigentlich nur ein paar Informationen darüber gehört.
            \item \interview{IV} Könntest du dir vorstellen einer der beiden gezeigten Darstellungsformen persönlich zu nutzen?
            \item \interview{P8} Wenn das bei mir auf dem Tablet schon eingestellt ist.
                  Dann beide.
                  Ich denke schon.
                  Also für mich persönlich, ja.
            \item \interview{IV} Hast du bereits mit Graphen gearbeitet?
            \item \interview{P8} Nein. Noch nicht.
            \item \interview{IV} Denkst du, dass die dir gezeigte neue Darstellungsform ein diverses Bild der Elektromobilität zeigt?
            \item \interview{P8} Ja, ich denke, das hilft viel.
            \item \interview{IV} Wie beeinflusst das Aussehen einer Webseite Deine Entscheidung, welche Informationen du liest und wie lange du auf der Seite bleibst?
            \item \interview{P8} Vermutlich stark, aber ich nutze mein Tablet nicht so oft.
            \item {------------------} [Demografische Daten] {------------------}
            \item \interview{P8} 70, weiblich, Realschule, Rentnerin
      \end{itemize}}
\nolinenumbers
