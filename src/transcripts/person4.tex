\section{Transkript - Person 4}
\sloppy
\texttt{\begin{itemize}[]
            \setlength\itemsep{0.02em}
            \resetlinenumber
            \linenumbers
            \item {-------------------------} [Intro] {-------------------------}
            \item {---------------------} [FBA - Tutorial] {---------------------}
            \item \interview{P4} Der Hase ist kein Gegenstand, sondern ein Lebewesen.
                  Spaß beiseite, aber das ist soweit verständlich.
            \item \interview{P4} Ist mit der Anzahl nur der Gegenstand Apfel gemeint, oder auch die anderen?
            \item \interview{IV} Auch die anderen Gegenstände.
            \item \interview{P4} Dann hätte ich zunächst drei die pflanzlich sind, aber der Rindfleischburger gehört anscheinend auch dazu.
                  Kannst du mir erklären, wieso das der Fall ist?
            \item \interview{IV} Wir hatten vorhin das Beispiel mit dem Schwein, welches sowohl Pflanzenfresser als auch Fleischfresser ist.
                  Beim Rindfleischburger ist es ähnlich.
                  Die Zusammensetzung des Brötchens ist pflanzlich, aber das Fleisch ist tierisch.
            \item \interview{P4} Alles klar. Dann hätte das vier und oben alle Nahrungsmittel wären sechs.
            \item \interview{P4} Hafer gehört zu Getreide, pflanzliche Nahrungsmittel und Nahrungsmittel.
            \item \interview{P4} Wenn ich das richtig verstanden habe, dann gehören all diese Artikel zu Staat und Industrie.
                  Die Zahlen verstehe ich, glaube ich, aber das ist trotzdem etwas kompliziert.
            \item {--------------------------} [FBA] {--------------------------}
            \item \interview{P4} Das hovern mit dem Mauszeiger finde ich noch etwas verwirrend.
                  Wo genau soll ich hovern?
            \item \interview{IV} Das ist nur in der Legende selbst möglich.
            \item \interview{P4} Achso. Das war mir nicht ganz klar.
            \item \interview{P4} Dann schau ich mich jetzt mal im Markt um.
                  Gut, dann gucken wir mal, ob sich ein E-Auto wirklich lohnt.
                  Also klicke ich mal auf den Artikel \flqq Stromverbrauch: So viel verbraucht ein E-Auto wirklich\frqq{}.
            \item \interview{P4} Hier sehe ich jetzt, dass der Artikel zu Staat und Industrie gehört.
            \item \interview{P4} Was ist denn hier passiert?
                  Ich habe auf Staat geklickt und sehe 0 Artikel.
                  Oh, moment ich glaube ich habe falsch gedrückt.
                  Aber dann würde ich jetzt \flqq Rohstoffe für E-Auto Akku\frqq{} klicken.
            \item \interview{IV} Kannst du deine Entscheidung für diesen Artikel noch kurz begründen?
            \item \interview{P4} Ja, weil Akkus offensichtlich sehr wichtig sind für Elektroautos.
                  Ich habe mal selbst zu diesem Thema recherchiert und gelesen wie umweltschädlich die Produktion von Akkus ist.
            \item \interview{P4} Dann würde ich auf Grün klicken. Damit ist Klima und Umwelt gemeint?
            \item \interview{IV} Genau. Also im Detail musst du die Werte hinter diesen Welten nicht verstehen, aber Grün steht hier für Umwelt.
            \item \interview{P4} Alles klar. Dann hätte ich das vielleicht auch so genannt.
                  Grün könnte man sonst einfach mit der politischen Partei verwechseln.
            \item \interview{P4} Als nächsten Artikel würde ich jetzt \flqq Neue Elektro-Studie unter der Lupe\frqq{} klicken.
                  Der Grund dafür ist, dass Studien oftmals gutes Wissen mit sich bringen und die Zukunft zeigen.
                  Zum Beispiel, welche Schäden die Produktion von E-Autos mit sich bringen kann.
            \item \interview{P4} Dann klicke ich mal auf Markt.
                  So ein E-Auto muss ja auch irgendwie bezahlt werden.
                  Und wie ich weiß sind die nicht gerade billig.
                  Also so mein aktueller Stand.
                  Da ist natürlich ein Artikel wie \flqq E-Auto Förderung 2021\frqq{} interessant.
            \item \interview{P4} Dann würde ich \flqq autonom. Fahren Erfolg oder Flop?\frqq{} klicken.
                  Das klingt eigentlich ganz interessant.
            \item \interview{IV} Kannst du das noch etwas näher beleuchten?
            \item \interview{P4} Einfach allgemein, um zu schauen, was die anderen zu diesem Thema denken.
                  Hier steht jetzt aber traurigerweise, dass bei diesem Artikel alles positiv ist.
                  Finde ich Schade, weil ich gerne auch negative Meinungen dazu in diesem Artikel gelesen hätte.
            \item \interview{IV} Du hast jetzt hauptsächlich nur mit den Kategorien gearbeitet.
                  Gibt es dafür einen bestimmten Grund?
            \item \interview{P4} Das ist eine sehr gute Frage.
                  Das weiß ich selber nicht so genau.
                  Aber ich denke mal, weil mir einfach mehr angezeigt wird.
                  Also einfach, weil die Übersicht besser ist.
            \item {---------------------} [Aufgaben - FBA] {---------------------}
            \item \interview{IV} Finde und klicke den Artikel \flqq Harter Schlag für Hersteller Plug-in Prämie fällt weg\frqq{}.
            \item \interview{P4} Hier würde ich dann genau den gerade genannten Vorteil ausnutzen.
                  Also ich kann einfach oben auf den Knoten klicken und sehe dann die ganze Übersicht.
            \item \interview{IV} Und dann jetzt noch mal den Artikel \flqq Deutlich sauberer als gedacht\frqq{}.
            \item \interview{P4} Hab ich.
            \item \interview{IV} Welcher Artikel ist am ähnlichsten zu \flqq Deutlich sauberer als gedacht\frqq{}?
            \item \interview{P4} Also ich würde sagen \flqq Von wegen nur das Klima retten\frqq{}.
            \item \interview{IV} Welcher Artikel unterscheidet sich am meisten von \flqq Deutlich sauberer als gedacht\frqq{}?
            \item \interview{P4} \flqq Rohstoffe für E-Auto Akkus\frqq{}, weil es da um die Akkus geht und das eigentlich negativ sein müsste.
            \item {----------------------} [QUESI - FBA] {----------------------}
            \item \interview{P4} Ich konnte auf eine mir vertraute Art mit dem System interagieren.
                  Lustige Frage.
                  Maus bewegen, klicken und lesen.
                  Schwierig ist das jetzt nicht.
            \item {-------------------} [AttrakDiff2 - FBA] {-------------------}
            \item {---------------------} [Listenansicht] {---------------------}
            \item \interview{P4} Hier würde ich \flqq Fürs Klima und gegen China\frqq{} klicken, weil gerne wissen will, warum man da gegen China sein kann.
                  Weil wenn wir Elektroautos und Akkus herstellen möchten sind wir auf China angewiesen.
                  Die haben ja die Rohstoffe.
            \item \interview{P4} \flqq Elektro-Zwang?\frqq{}.
                  Hier habe ich einfach nur draufgeklickt, weil ich wissen möchte welcher Zwang gemeint ist.
                  Vermutlich Clickbait.
            \item \interview{P4} \flqq Autobauer als Software-Riesen\frqq{} würde danach kommen, weil in der heutigen Zeit Software immer wichtiger wird.
                  In Zukunft werden sowieso alle Betriebe Software-Betriebe werden.
                  Das sind man bereits in der Lebensmittelindustrie und deren Apps.
                  Aber auch die Systeme im Hintergrund sind ja bereits Softwaregetrieben.
            \item \interview{P4} \flqq Jetzt kommen die Wunderakkus\frqq{}.
                  Ich möchte wissen was genau verbessert wird.
                  Speziell das Recycling von Akkus würde mich interessieren.
            \item \interview{P4} Den Artikel \flqq Strom, Trassen, Verteilernetze\frqq{} würde ich klicken, weil wenn man ein E-Auto hat, muss man auch schauen, wo man hinkommt und wo Lademöglichkeiten sind.
            \item {----------------} [Aufgaben - Listenansicht] {----------------}
            \item \interview{IV} Finde und klicke den Artikel \flqq Strom, Trassen, Verteilernetze\frqq{}.
            \item \interview{P4} Das ist ziemlich einfach.
            \item \interview{IV} Finde und klicke den Artikel \flqq Autobauer als Software-Riesen\frqq{}.
            \item \interview{P4} Yup.
            \item \interview{IV} Welcher Artikel ist am ähnlichsten zu \flqq Autobauer als Software-Riesen\frqq{}?
            \item \interview{P4} Das hier.
            \item \interview{IV} Also \flqq Auto-Bloggerin nimmt Tesla-Modell auseinander\frqq{}.
            \item \interview{IV} Welcher Artikel unterscheidet sich am meisten von \flqq Autobauer als Software-Riesen\frqq{}?
            \item \interview{P4} Das ist schwierig.
                  Brain.exe stürzt ab.
                  Puh.
                  Weiß nicht.
                  Ich würde jetzt einfach sagen Diess kontert Reitzle.
            \item \interview{IV} Und du entscheidest dich dafür mit einem Grund oder jetzt einfach so aus dem Bauch heraus?
            \item \interview{P4} Eher aus dem Bauch heraus.
                  Bei einem ähnlichen Artikel hätte ich noch eine Idee, weil Tesla und Software-Riese thematisch passen.
                  Aber bei einem unterschiedlichen Artikel bin ich eher verwirrt.
            \item {-----------------} [QUESI - Listenansicht] {-----------------}
            \item \interview{P4} Hier ist jetzt die Frage: Zählt das als Problem, dass ich das Gegenteil des vorherigen Artikels nicht kannte?
            \item \interview{IV} Es geht eher um deinen Umgang mit dem System und das intuitive Verständnis dazu.
                  du kannst es mit einfließen lassen, wenn du es für ein relevantes Problem hältst.
            \item {--------------} [AttrakDiff2 - Listenansicht] {--------------}
            \item {-----------------------} [Fragebogen] {----------------------}
            \item \interview{IV} Welche Darstellungsformen für Online-News-Artikel sind dir bekannt?
            \item \interview{P4} Nicht wirklich.
            \item \interview{IV} Wie häufig besuchst du News-Webseiten?
            \item \interview{P4} Aktuell gar nicht.
                  Ich kriege die meisten Informationen nebenbei über TV mit, wenn Nachrichten laufen.
                  Aber ansonsten juckt mich das herzlich wenig, weil alle nur Blödsinn schreiben, um Klicks zu bekommen.
            \item \interview{IV} Nur aus Neugier, weil das möglicherweise auch relevant sein kann: Wie häufig schaust du dir Nachrichten im TV an?
            \item \interview{P4} Vielleicht alle zwei, drei Tage. Im Grunde genommen ändert sich ja nicht viel.
                  Und bei News-Webseiten könnte ich vielleicht sagen einmal pro Monat.
                  Ab und zu klicke ich nämlich auf Artikel auf meinem Handy, wenn Google irgendetwas vorschlägt.
            \item \interview{IV} Inwieweit hast du bereits Vorerfahrung zum Thema E-Mobilität?
            \item \interview{P4} Etwas länger her, aber ich hatte da öfters mal mit einem Kumpel darüber diskutiert.
                  Also vielleicht etwas reingeschnuppert in das Thema.
            \item \interview{IV} Könntest du dir vorstellen einer der beiden gezeigten Darstellungsformen persönlich zu nutzen?
            \item \interview{P4} Also im täglichen Gebrauch oder wie?
            \item \interview{IV} In deinem persönlichen Gebrauch einfach.
                  Wie oft das ist, kannst du entscheiden.
                  Wenn du die Seite nur einmal pro Monat besuchen möchtest, um dir Artikel durchzulesen ist das auch in Ordnung.
            \item \interview{P4} Kommt darauf an. Ich fände es eigentlich nicht schlecht, wenn man beide verbinden würde.
                  Eine Liste an Themen, welche es gibt zum Beispiel zum Thema Elektronik.
                  Da kann man dann auf Dinge wie Autos, Computer oder E-Mobilität klicken.
                  Und wenn man darauf geklickt hat, kommt dieser Graph.
            \item \interview{IV} Das ginge in der Theorie auch mit einem Oberbegriffsgraph, kannst du dir das vorstellen?
            \item \interview{P4} Vorstellen kann ich es mir, aber ich weiß nicht, ob ich das nicht zu kompliziert finden würde.
            \item \interview{IV} Alles klar, aber um nochmal auf die Frage zurückzukommen.
                  Könntest du dir vorstellen einer der beiden gezeigten Darstellungsformen persönlich zu nutzen?
            \item \interview{P4} Ja, beide.
            \item \interview{IV} Hast du bereits mit Graphen gearbeitet?
            \item \interview{P4} Ja, in diversen Videospielen.
            \item \interview{IV} Denkst du, dass die dir gezeigte neue Darstellungsform ein diverses Bild der Elektromobilität zeigt?
            \item \interview{P4} Also es gibt dadurch eine Übersicht.
                  Man hat direkt Artikel kategorisiert.
                  Wenn ich also sehe, dass ein Knoten mit Staat und Markt zu tun hat, kann ich mir eher etwas darunter vorstellen als, wenn ich nur einen Titel sehe.
                  Man sieht gut wie Artikel zusammenhängen und diese Möglichkeit gibt es bei der Liste nicht.
            \item \interview{IV} Wie beeinflusst das Aussehen einer Webseite Deine Entscheidung, welche Informationen du liest und wie lange du auf der Seite bleibst?
            \item \interview{P4} Sehr viel.
                  Wenn eine Website für mich unseriös aussieht, dann klicke ich gar nicht erst darauf.
            \item {------------------} [Demografische Daten] {------------------}
            \item \interview{P1} 30, männlich, Fachhochschule, Selbstständig im Bereich 3D-Modellierung und Druck
      \end{itemize}}
\nolinenumbers
