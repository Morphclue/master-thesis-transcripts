\section{Transkript - Person 2}
\sloppy
\texttt{\begin{itemize}[]
        \setlength\itemsep{0.02em}
        \linenumbers
        \item {-------------------------} [Intro] {-------------------------}
        \item {---------------------} [FBA - Tutorial] {---------------------}
        \item \interview{P2} Ich würde hier einfach den Hasen tippen.
        \item \interview{P2} Okay, das heißt, das wird jetzt hochgezählt.
        \item \interview{P2} Eine Kategorie, welche sich unter Säugetier einordnen lässt.
              Auto macht wohl keinen Sinn.
              Buchstabe leider auch nicht.
              Also ja, kommt natürlich auf den Kontext drauf an.
              Ja, man könnte die Säugetiere anhand der Buchstaben sortieren, ist aber irgendwie jetzt von der Semantik her irgendwie nicht wirklich zielführend.
              Von daher dann eben Fleischfresser, weil das irgendwie mehr dazu zählt.
              Also wie gesagt thematisch besser passt.
        \item \interview{IV} An dieser Stelle kannst du mir einfach laut sagen, was du da als Lösung auf die Fragen hast.
        \item \interview{P2} Theoretisch kannst du das ja nicht doppelt zählen. Das ist jetzt etwas tricky.
        \item \interview{P2} Hier sind es drei.
              Mit dem Apfel zusammen sind es hier auf der Seite vier, also hier kommt eine vier hin.
              Und wenn das vier sind und hier steht schon die zwei.
              Dann haben wir hier oben vier plus zwei sind sechs.
        \item \interview{P2} Hafer hat unter Getreide gelistet, also Kategorie Getreide.
              Also Hafer zählt zur untersten Kategorie.
              Getreide aber ist auch inkludiert in pflanzliche Nahrungsmittel und in Nahrungsmittel.
              Selbstverständlich, aber in Gegenteil nicht tierische Nahrungsmittel.
              Ja, und auch nicht zu der Kategorie ohne Betitelung.
        \item \interview{P2} Jeder Artikel kann positiv, negativ, okay, das heißt es wird mit Color-coding dargestellt. Finde ich gut.
        \item \interview{P2} Okay, deswegen wird von der Kategorie plus eins gerechnet.
              Okay, ist mir nicht so ganz bewusst, was das bedeuten soll.
              Das finde ich am schwierigsten zu verstehen.
              Ich will es mir trotzdem noch mal ganz kurz durchlesen, vielleicht wird es klarer, wenn ich es nochmal lese.
              Die Liste enthält alle Artikel, welche zur ausgewählten Kategorie passen.
              Für jede weitere Kategorie wird in der Liste plus eins gerechnet.
              Also eins sind dann alle, die hier sind.
              Warum ich jetzt hier auf die zwei kam, war eben genau, weil ich dann quasi plus eins plus noch mal eins gehe, weil ich das ja noch als Knoten sehe.
              Der hat zwar keine Betitelung, aber deswegen hätte ich da jetzt irgendwie so intuitiv noch eine zwei vermutet.
        \item {--------------------------} [FBA] {--------------------------}
        \item \interview{P2} Hier haben wir das Gesamtkonstrukt dann anscheinend.
              Also der Knoten ganz oben.
              Also man könnte den auch einfach als E-Mobilität betiteln.
        \item \interview{P2} Markt, Staat, Industrie und Grün.
              Und darunter sind das dann jetzt logischerweise wieder die Artikel, die verschiedenen nicht benannten Kategorien zugeordnet wurden.
        \item \interview{P2} Ich würde jetzt hier einfach mal den untersten Artikel anschauen. Muss ich oder soll ich erklären warum?
        \item \interview{IV} Ja. Das wäre super, wenn du deine Gedanken dazu auch noch mal kurz erläuterst.
        \item \interview{P2} Okay, also tatsächlich \flqq E-Auto Förderung 2021\frqq{} klicke ich nicht an, weil es zu sehr in der Vergangenheit liegt.
              \flqq Neue Elektro-Studie unter der Lupe\frqq{} - Ja, weiß ich nicht, ob das jetzt so interessant für mich ist in dem Sinne.
              \flqq Neue E-Auto-Förderung Tesla\frqq{} klicke ich nicht, weil es mir zu spezifisch ist.
              \flqq Von wegen nur das Klima retten\frqq{} nicht, weil es nicht wirklich nach einem wissenschaftlichen oder vernünftigen Titel klingt.
              \flqq Diesel soll 20 Cent teurer werden\frqq{} - Betrifft mich nicht, weil ich kein Diesel fahre.
              \flqq Rohstoffe für E-Auto Akku\frqq{} - Da gibt es ja immer verschiedene Ansichten für die Rohstoffe von so einem E-Auto Akku.
              Von daher hätte ich das jetzt einfach angeklickt, weil es halt am interessantesten klingt.
        \item \interview{P2} Gut okay, das heißt in diesem Text sind jetzt verschiedene Aspekte drinnen, die ich nicht bewerte, sondern die schon bewertet wurden.
              Und die sind jetzt hier zum Beispiel beim Staat negativ.
              Aber warum sind die jetzt bei Grün positiv?
              Also ich hätte jetzt vermutet, dass es wahrscheinlich ein kritischer Artikel ist, aber ist wahrscheinlich auch nicht so relevant gerade, oder?
        \item \interview{IV} In dem Fall würde positiv Grün bedeuten, dass dieser Artikel positiv aus der grünen Welt heraus argumentiert.
              Die genaue Klassifizierung dafür ist hier nicht so wichtig.
        \item \interview{P2} Okay, in Ordnung, dann gehe ich wieder zurück.
        \item \interview{P2} Und wenn ich das anklicke?
              Also ich weiß nicht, ob das vielleicht hilfreich ist, aber ich finde es interessanter, die Artikel-Ansicht anzuklicken, um zu sehen, welche Artikel jetzt zu diesem Knoten gehören und auf was sie Einfluss haben.
              Das ist, finde ich zumindest, interessanter als am Ende mir diese Gesamtübersicht anzugucken.
              Liegt aber auch daran, dass ich von Bottom zu Top angefangen habe und habe mir hier die Artikel anschaue.
              Das ist so der Grund, warum ich jetzt eher dann auf die unteren Hälften klicke und mich nach oben navigieren.
              Vielleicht auch nicht ganz intuitiv.
              Ich würde mir \flqq autonom. Fahren Erfolg oder Flop?\frqq{} noch durchlesen.
        \item \interview{P2} Also gut. Jetzt habe ich noch mal das oben angeklickt, damit ich die Berichte mehr angucken kann.
              Was natürlich mega top ist, dass man halt, alle Artikel sieht, die auch da drunter sind.
              Also es hat schon einen deutlichen Vorteil, wenn man die Kategorie anklickt.
              Die klicke ich jetzt auf jeden Fall mal beide Artikel an.
              \flqq Stromverbrauch: So viel verbraucht ein E-Auto wirklich\frqq{} und \flqq Lohnt sich ein E-Auto bei den Strompreisen noch\frqq{}.
        \item \interview{P2} Und zuletzt vielleicht \flqq Revolution oder Wegwerfauto\frqq{}.
        \item \interview{IV} Ich habe generell so ein kleines bisschen was aufnehmen können, auch das zu den Kategorien, was ich eigentlich ganz interessant fand.
              Also, dass du beschrieben hast, dass du eher diese Bottom to Top Ansicht bevorzugst, also, dass du eher auf die Artikel klickst.
              Später hast du dann aber auch die Top to Bottom Ansicht ausprobiert.
              Kannst du mir dazu vielleicht noch etwas erzählen?
        \item \interview{P2} Also, letzten Endes ist diese Kategorie-Ansicht, glaube ich, schon cooler als sich die einzelnen Artikel für die verschiedenen Knoten anzuschauen.
              Wenn ich jetzt so drüber nachdenke, finde ich sogar die Ansicht mit den Kategorien wahrscheinlich doch besser, weil wenn halt alles dazu gelistet wird, was dazuzählt.
              Und dann hat man halt auch noch mal so eine Untergliederung in, ich sage mal in detailliertere Sachen halt. Also related Artikel sage ich jetzt einfach mal.
              Ich glaube, das, was mich am Anfang so ein bisschen davon abgeschreckt hat, die Kategorien zu wählen ist erst mal die Graphen-Ansicht wahrscheinlich.
              Das wäre für mich intuitiver, wenn es wie bei einem File-Explorer wäre.
              Also man hat einen Ordner, der über einem Ordner steht und dann hat man Unterordner.
              Wenn es so gewesen wäre wette ich mit dir, dass ich vermutlich die Kategorien-Ansicht eher genutzt hätte.
              Aber ansonsten finde ich das ziemlich nice, dass man die Sachen, also die Kategorie anklicken kann und man hat alles gelistet bekommt.
        \item \interview{P2} Dadurch, dass es jetzt hier noch alles unzählbare Berichte sind, ginge das noch, wenn man direkt von oben anfängt.
              Aber ich habe jetzt halt von unten angefangen, weil ich noch nicht wusste, wie viele Artikel so darunter sind.
              Das können ja prinzipiell auch Graphen sein, die recht groß sind und dann wird es halt total unübersichtlich am Anfang.
              Deswegen fand ich es ganz gut von unten nach oben zu arbeiten.
        \item {---------------------} [Aufgaben - FBA] {---------------------}
        \item \interview{IV} Finde und klicke den Artikel \flqq Harter Schlag für Hersteller Plugin Prämie fällt weg\frqq{}.
              Okay. Wird wahrscheinlich auf der linken Seite sein, weil es Staat betrifft und die Industrie und den Markt wahrscheinlich auch indirekt.
              Aber ich glaube, das wird wahrscheinlich hierunter gelistet sein.
              Hoffe ich jetzt zumindest.
              Nö, eben nicht.
              Unlucky.
              Also ich wollte es jetzt so spezifisch wie möglich machen.
              Ich könnte natürlich auch hier lang, also den Knoten anklicken und dann schauen, wo der Artikel ist.
        \item \interview{IV} Also du hast den Knoten angeklickt, welchen du am Anfang als E-Mobilität bezeichnet hast?
        \item \interview{P2} Ja, genau.
              Also wenn man tatsächlich den Artikel nur sucht, wäre die klügere Wahl gewesen, direkt über die große Kategorie zu gehen.
              Aber ich habe es jetzt halt eben so gemacht, dass ich das irgendwie probiert habe, den Artikel mit den Kategorien zu verbinden, wo ich den Artikel zuordnen würde.
        \item \interview{IV} Und dann jetzt noch mal den Artikel \flqq Deutlich sauberer als gedacht\frqq{}.
        \item \interview{P2} Würde ich jetzt Grün zuordnen. Also hier.
        \item \interview{IV} Welcher Artikel ist am ähnlichsten zu \flqq Deutlich sauberer als gedacht?\frqq{}?
        \item \interview{P2} Ähm, auf jeden Fall einen Artikel, der dieser Kategorie am nächsten ist.
              Also natürlich einer in derselben Kategorie.
              Dadurch, dass wir jetzt aber hier nur diesen deutlich sauberer als gedacht haben - dann eben in der nächst unteren Stufe.
              Also \flqq Harter Schlag für Hersteller Plug-in Prämie fällt weg\frqq{}, würde ich jetzt intuitiv sagen.
        \item \interview{IV} Welcher Artikel unterscheidet sich am meisten von \flqq Deutlich sauberer als gedacht?\frqq{}?
        \item \interview{P2} Nun ja, wenn wir das Gesamtkonstrukt ansehen, irgendetwas, was nicht mit diesem Knoten in Verbindung steht, sprich nichts, was da drunter liegt.
              Also nur Sachen, die auf die keine Verbindung nach unten oder nach oben haben.
              Und wenn es sowas nicht gibt, dann den Knoten, welcher am weitesten entfernt liegt.
              Also zum Beispiel der Artikel \flqq autonom. Fahren Erfolg oder Flop?\frqq{}.
        \item {----------------------} [QUESI - FBA] {----------------------}
        \item \interview{P2} Könnte ich hier noch etwas dazu sagen?
        \item \interview{IV} Ja klar. Also sobald dir etwas einfällt, kannst du es gerne sagen. Ich freue mich über jeden zusätzlichen Input.
        \item \interview{P2} Also das Tutorial hat geholfen zu verstehen, wie die Kategorien mit den darunterliegenden Artikeln strukturiert werden und auch miteinander zusammenhängen.
        Hätte man das Tutorial jetzt am Anfang nicht gehabt, dann wäre das tatsächlich so ein bisschen Trial-and-Error und dann hätte man glaube ich auch nicht direkt gewusst, warum irgendwelche Stränge jetzt markiert werden, wenn man auf Artikel oder Kategorien klickt.
        \item \interview{P2} Okay, also man musste natürlich erst mal reinkommen, aber dann war es eigentlich recht intuitiv, nachdem man halt tatsächlich mal ein bisschen was damit gemacht hat.
              Nur das dann anzuwenden und zu sehen, wie es genau in der Praxis funktioniert, war dann ein bisschen tricky.
              Ja, wobei, es wurde ja direkt vernünftig erklärt.
              Das heißt, es war mir schon klar, wie es funktioniert.
              Also wenn man das Tutorial mitzählen soll, dann schon.
        \item \interview{IV} Ja genau. Das Tutorial ist mit einbegriffen in diesem und dem folgenden Fragebogen.
        \item {-------------------} [AttrakDiff2 - FBA] {-------------------}
        \item {---------------------} [Listenansicht] {---------------------}
        \item \interview{P2} Okay, also ich sehe eine Liste, aber die ist erstmal relativ zusammenhangslos.
              Also auch was Kategorien angeht.
        \item \interview{P2} \flqq Diess kontert Reitzle\frqq{} interessiert mich schonmal nicht, weil ich die Namen nicht kenne.
              \flqq 30.000 Euro geschenkt für einen Tesla\frqq{} auch nicht, weil es wieder zu spezifisch ist.
              \flqq Willkommen in der grünen Flammenhölle\frqq{} auch nicht, weil der Titel nichts aussagend ist.
              Und \flqq Auto-Bloggerin nimmt Tesla-Modell auseinander\frqq{} auch nicht wirklich.
        \item \interview{P2} Dann würde ich jetzt von oben nach unten durchgehen und schauen, was mich interessiert.
        \item \interview{P2} Also \flqq Elektro Zwang?\frqq{} klingt vielleicht ganz gut, weil das ja recht gut gefördert wird.
        \item \interview{P2} \flqq Fürs Klima und gegen China\frqq{} würde ich jetzt auf jeden Fall auch lesen.
              Also \flqq Fürs Klima und gegen China\frqq{} würde ich mir durchlesen, weil ich das gut finde, wenn wir halt vielleicht nicht so abhängig sind von anderen Staaten, was E-Mobilität angeht.
        \item \interview{P2} Ich würde wieder von oben nach unten weiter durchgehen.
              \flqq Autobauer als Software-Riesen\frqq{}.
              Dadurch, dass ich Informatiker bin, habe ich eben auch ein gewisses Interesse daran, welche Rolle Software bei den Autobauern spielt.
        \item \interview{P2} \flqq Umverteilung von Arm nach Grün\frqq{} ist jetzt irgendwie nichts für mich. Würde ich mir jetzt anhand des Titels nicht durchlesen wollen.
              \flqq Strom, Trassen, Verteilernetze\frqq{} auch nicht.
              \flqq Jetzt kommen die Wunderakkus\frqq{} schon eher.
              Das klingt aber nach einem Clickbait, bin ich ehrlich.
        \item \interview{P2} Und tatsächlich würde ich jetzt auch nicht viel mehr von diesen Artikeln lesen wollen.
        \item \interview{IV} Könntest Du vielleicht noch einen weiteren Artikel nennen? Also auch, wenn dieser dich nicht 100\%ig interessiert?
        \item \interview{P2} Ja, dann würde ich mir wahrscheinlich \flqq Strom, Trassen, Verteilernetze\frqq{} durchlesen.
              Und auch genau in dieser Reihenfolge, die ich jetzt gerade hatte.
              Also das wäre tatsächlich so von most interesting für mich zu least interesting.
        \item {---------------} [Aufgaben - Listenansicht] {---------------}
        \item \interview{IV} Finde und klicke den Artikel \flqq Strom, Trassen, Verteilernetze\frqq{}
        \item \interview{P2} Okay, ganz einfach.
        \item \interview{IV} Finde und klicke den Artikel \flqq Autobauer als Software-Riesen\frqq{}
        \item \interview{P2} Yep, hab ich.
        \item \interview{IV} Welcher Artikel ist am ähnlichsten zu \flqq Autobauer als Software-Riesen\frqq{}
        \item \interview{P2} Da ich bereits gemerkt habe, dass die Artikel rechts in keiner wirklichen Ordnung sind müsste ich anhand der Titel das ganze herausfinden.
        Hier würde ich den Artikel \flqq Herstellungskosten gecheckt\frqq{} nehmen, da er auch die Sicht der Autobauer beleuchtet.
        \item \interview{IV} Welcher Artikel unterscheidet sich am meisten von \flqq Autobauer als Software-Riesen\frqq{}.
        \item \interview{P2} Am unterschiedlichsten würde ich \flqq Auto-Bloggerin nimmt Tesla-Modell auseinander\frqq{} vermuten, da der Titel thematisch erst einmal nicht unbedingt Gemeinsamkeiten aufweist.
        \item {-----------------} [QUESI - Listenansicht] {-----------------}
        \item \interview{P2} Bei der Listenansicht haben mir irgendwie die Kategorien gefehlt.
              Da fande ich das mit der Kategorisierung schon besser.
              Aber es war trotzdem einfach zu verwenden.
        \item {--------------} [AttrakDiff2 - Listenansicht] {--------------}
        \item \interview{IV} Welche Darstellungsformen für Online-News-Artikel sind Dir bekannt?
        \item \interview{P2} Einmal die Listenansicht ganz klassisch auf der linken oder rechten Seite.
        Dann gibt es ja ganz oft diese, was man bei web.de oder GMX findet, also eine Art Kachelansicht.
        Und dann halt im mobilen Bereich einen Endless-Scroll.
        Also wie bei Google News oder Instagram.
        \item \interview{IV} Wie häufig besuchst Du News-Webseiten?
        \item \interview{P2} Ich würde sagen zweimal pro Tag vielleicht. Aber meistens über mobile Geräte.
        \item \interview{IV} Inwieweit hast du bereits Vorerfahrung zum Thema E-Mobilität?
        \item \interview{P2} Ja, etwas. Zwar nicht sonderlich viel, aber so ein gewisses ähm, ja, so ein paar Sachen habe ich mir schon durchgelesen.
        \item \interview{IV} Könntest Du Dir vorstellen einer der beiden gezeigten Darstellungsformen persönlich zu nutzen?
        \item \interview{P2} Ja, definitiv. Zum einen diese Listenansicht ist ja tatsächlich relativ weit verbreitet.
        Wenn auch manchmal in abgewandelter Form.
        Also ich fand das mit diesen Graphen tatsächlich recht interessant.
        Das müsste aber halt noch visuell irgendwie ein bisschen mehr zugänglich gemacht werden, denke ich.
        Aber ansonsten finde ich das ziemlich cool, dass man halt über diese Graphen recht schnell zu irgendwelchen Artikeln kommt.
        Ja, und wenn die Knoten, die noch keine Betitelung hatten, wenn die sozusagen auch noch eine Betitelung bekommen würden, dann wäre die Navigation durch diesen ganzen Graphen noch intuitiver und einfacher.
        \item \interview{IV} Hast Du bereits mit Graphen gearbeitet?
        \item \interview{P2} Ja, nicht oft, aber habe ich schon.
        \item \interview{IV} Denkst Du, dass die Dir gezeigte neue Darstellungsform ein diverses Bild der Elektromobilität zeigt?
        \item \interview{P2} Was diese Ansicht halt extrem cool macht, ist halt in diese Kategorien aufzuteilen.
        Und was auch cool ist, ist, dass durch dieses Color-coding mit grün, gelb und rot, dass man halt auch schon direkt sozusagen die Einstellung sieht, wie der Artikel gegenüber der Thematik eingestellt ist oder verfasst wurde. 
        Das heißt, ich würde jetzt die Frage mit Ja beantworten.
        \item \interview{IV} Wie beeinflusst das Aussehen einer Webseite Deine Entscheidung, welche Informationen Du liest und wie lange Du auf der Seite bleibst?
        \item \interview{P2} Tatsächlich würde ich sagen sehr stark, denn gut aussehende Webseiten sprechen mich mehr an als konventionelle Webseiten.
        \item {------------------} [Demografische Daten] {------------------}
        \item \interview{P2} 28, männlich, Master, Informatiker
    \end{itemize}}
\nolinenumbers
