\section{Transkript - Person 7}
\sloppy
\texttt{\begin{itemize}[]
            \setlength\itemsep{0.02em}
            \resetlinenumber
            \linenumbers
            \item {-------------------------} [Intro] {-------------------------}
            \item {---------------------} [Listenansicht] {---------------------}
            \item \interview{IV} Du hast jetzt \flqq Fürs Klima und gegen China\frqq{} geklickt.
                  Gibt es einen bestimmten Grund, weswegen du darauf geklickt hast?
            \item \interview{P7} Ich finde die anderen Artikel etwas komplizierter.
                  Bei diesem Artikel habe ich das Gefühl, dass ich den Inhalt besser verstehe.
            \item \interview{IV} Jetzt hast du \flqq Umverteilung von Arm nach Grün\frqq{} geklickt. Kannst du auch hier wieder einen Grund nennen, weswegen du darauf geklickt hast?
            \item \interview{P7} Ich kann es dir echt nicht sagen.
                  Ich habe mir jetzt das herausgesucht, was mich mehr oder weniger interessiert.
            \item \interview{IV} Was genau fandest du bei diesem Artikel interessant?
            \item \interview{P7} Einfach, weil es ein aktuelles Thema ist.
                  Es kommt auch hin und wieder in den Medien vor, weswegen ich da jetzt auch drauf geklickt habe.
            \item \interview{P7} \flqq Herstellungskosten gecheckt\frqq{} klingt vielleicht sinnvoll, wenn ich mich zur E-Mobilität informieren möchte.
            \item \interview{P7} \flqq Elektro-Zwang?\frqq{} klingt für mich eher negativ.
                  Da würde ich aber trotzdem drauf klicken, weil ich natürlich nicht zu etwas gezwungen werden möchte.
            \item \interview{P7} Passend zum Thema Klima und Umverteilung von Arm nach Grün würde ich jetzt vielleicht noch \flqq Willkommen in der grünen Flammenhölle\frqq{} klicken.
            \item \interview{IV} Und weswegen möchtest du dich so intensiv mit den Grünen beschäftigen?
            \item \interview{P7} Ich finde einfach, dass die Grünen recht kontrovers sind.
                  Die spielen im Thema E-Mobilität auch eine große Rolle, würde ich jetzt mal so behaupten.
            \item {----------------} [Aufgaben - Listenansicht] {----------------}
            \item \interview{IV} Finde und klicke den Artikel \flqq Strom, Trassen, Verteilernetze\frqq{}.
            \item \interview{P7} So, wo war denn das? Hier.
            \item \interview{IV} Finde und klicke den Artikel \flqq Autobauer als Software-Riesen\frqq{}.
            \item \interview{P7} Hier.
            \item \interview{IV} Welcher Artikel ist am ähnlichsten zu \flqq Autobauer als Software-Riesen\frqq{}?
            \item \interview{P7} Das ist aber eine schwierige Frage. Ich würde sagen, nehmen wir doch einfach \flqq Elektro-Zwang?\frqq{}.
            \item \interview{IV} Welcher Artikel unterscheidet sich am meisten von \flqq Autobauer als Software-Riesen\frqq{}?
            \item \interview{P7} Das ist schwierig. Für mich sind alle Artikel irgendwie anders. Dann vielleicht \flqq Strom, Trassen, Verteilernetze\frqq{}.
            \item {-----------------} [QUESI - Listenansicht] {-----------------}
            \item {--------------} [AttrakDiff2 - Listenansicht] {--------------}
            \item {---------------------} [FBA - Tutorial] {---------------------}
            \item \interview{P7} Für pflanzliche Nahrungsmittel hätte ich jetzt gesagt, dass es drei wären.
                  Der Rindfleischburger ist zwar noch darunter, aber nicht pflanzlich.
            \item \interview{IV} Der Rindfleischburger ist wie das Schwein sowohl pflanzlich als auch tierisch. Das Brötchen ist pflanzlich und das Fleisch ist tierisch.
            \item \interview{P7} Dann hätten wir vier pflanzliche Nahrungsmittel. Und für die Nahrungsmittel wären es dann fünf.
            \item \interview{P7} Kategorien für Hafer wären pflanzliche Nahrungsmittel, Nahrungsmittel und Getreide.
            \item \interview{P7} Das wird ja immer komplizierter.
            \item \interview{P7} Okay, aber ich glaube ich habe das verstanden.
            \item {--------------------------} [FBA] {--------------------------}
            \item \interview{P7} Dann würde ich hier mir jetzt auch zunächst diesen Grünen Knoten anschauen.
                  \flqq Von wegen nur das Klima retten\frqq{} klingt doch spannend.
            \item \interview{P7} Dann gehen wir jetzt zu Staat. \flqq autonom. Fahren Erfolg oder Flop?\frqq{}
            \item \interview{IV} Gibt es einen bestimmten Grund, weswegen du diesen Artikel geklickt hast?
            \item \interview{P7} Ich finde das Thema einfach interessant.
                  Ich habe aktuell noch keine Meinung dazu oder weiß auch nicht wie das funktioniert.
                  Deswegen wollte ich mir diesen Artikel durchlesen.
                  Vielleicht auch in Hinsicht darauf, wie sich das über die Jahre entwickelt.
            \item \interview{P7} \flqq Rohstoffe für E-Auto Akku\frqq{} klingt auch interessant.
                  Den Artikel würde ich mir auch noch anschauen.
                  Hier unten in der Kategorie scheint es auch negativ bei Staat zu sein und hat auch etwas mit der Grün-Kategorie zu tun.
                  Man muss ja auch schauen, woher die Rohstoffe kommen und das ist generell alles irgendwie zusammenhängend.
            \item \interview{P7} Jetzt können wir uns auch etwas anderes anschauen.
                  Gehen wir mal auf Markt.
                  \flqq Lohnt sich E-Auto bei den Strompreisen noch?\frqq{}.
                  Das ist natürlich eine sehr gute Frage und auch aktuell.
                  Wenn einer das durchrechnen würde, würde mich das interessieren.
            \item \interview{P7} \flqq Neue Elektro-Studie unter der Lupe\frqq{} interessiert mich, weil ich denke, dass Studien viele Perspektiven zeigen.
                  Also einfach vielfältig sind.
                  Sie zeigen vielleicht die Meinung von verschiedenen Leuten und Experten.
                  Und hier sehen wir auch, dass es sowohl positiv als auch negativ ist.
            \item {---------------------} [Aufgaben - FBA] {---------------------}
            \item \interview{IV} Finde und klicke den Artikel \flqq Harter Schlag für Hersteller Plug-in Prämie fällt weg\frqq{}.
            \item \interview{P7} Nehmen wir erstmal Staat. Alles klar, ich habe den Artikel gefunden.
            \item \interview{IV} Und dann jetzt noch mal den Artikel \flqq Deutlich sauberer als gedacht\frqq{}.
            \item \interview{P7} Das würde ich jetzt unter Grün vermuten.
            \item \interview{IV} Welcher Artikel ist am ähnlichsten zu \flqq Deutlich sauberer als gedacht\frqq{}?
            \item \interview{P7} Eigentlich auch bei Grün etwas.
                  Aber thematisch würde ich vielleicht \flqq Stromverbrauch: So viel verbraucht ein E-Auto wirklich\frqq{} nehmen.
            \item \interview{IV} Welcher Artikel unterscheidet sich am meisten von \flqq Deutlich sauberer als gedacht\frqq{}?
            \item \interview{P7} \flqq Von wegen nur das Klima retten\frqq{}.
            \item {----------------------} [QUESI - FBA] {----------------------}
            \item {-------------------} [AttrakDiff2 - FBA] {-------------------}
            \item {-----------------------} [Fragebogen] {----------------------}
            \item \interview{IV} Welche Darstellungsformen für Online-News-Artikel sind dir bekannt?
            \item \interview{P7} Ich kenne keine weiteren Darstellungsformen.
            \item \interview{IV} Wie häufig besuchst du News-Webseiten?
            \item \interview{P7} Zweimal pro Woche.
            \item \interview{IV} Inwieweit hast du bereits Vorerfahrung zum Thema E-Mobilität?
            \item \interview{P7} Gar keine.
            \item \interview{IV} Könntest du dir vorstellen einer der beiden gezeigten Darstellungsformen persönlich zu nutzen?
            \item \interview{P7} Beide eigentlich.
            \item \interview{IV} Kannst du auch erklären wieso? Also du kannst gerne mit der Liste dann anfangen.
            \item \interview{P7} Weil die eigentlich schon bekannt sind.
                  Also ich nutze die Liste ja bereits persönlich.
                  Und beim Graphen denke ich, dass man sich einfach Zeit lassen muss, um den zu verstehen.
                  Dann, nachdem man sich eingearbeitet hat, denke ich, dass der Graph auch sehr gut ist.
            \item \interview{IV} Hast du bereits mit Graphen gearbeitet?
            \item \interview{P7} Nein.
            \item \interview{IV} Denkst du, dass die dir gezeigte neue Darstellungsform ein diverses Bild der Elektromobilität zeigt?
            \item \interview{P7} Du meinst, dass man diese Vielfalt darstellen kann?
            \item \interview{IV} Ja genau.
            \item \interview{P7} Würde ich schon sagen.
            \item \interview{IV} Wie beeinflusst das Aussehen einer Webseite Deine Entscheidung, welche Informationen du liest und wie lange du auf der Seite bleibst?
            \item \interview{P7} Sehr.
                  Also ich finde es wichtig, dass die Website nicht schlecht aussieht oder überladen ist.
                  Bei den Prototypen denke ich mal, dass die noch schöner gemacht werden, oder?
            \item \interview{IV} Ja, genau. Also die Prototypen sind sozusagen erstmal nur da, um das Konzept beziehungsweise das System zu testen.
            \item {------------------} [Demografische Daten] {------------------}
            \item \interview{P1} 50, weiblich, Abitur, Augenoptikerin
      \end{itemize}}
\nolinenumbers
