\section{Transkript - Person 9}
\sloppy
\texttt{\begin{itemize}[]
            \setlength\itemsep{0.02em}
            \linenumbers
            \item {-------------------------} [Intro] {-------------------------}
            \item {---------------------} [Listenansicht] {---------------------}
            \item \interview{P9} \flqq Fürs Klima und gegen China\frqq{} springt mir sofort ins Auge, weil ich mir nicht vorstellen kann, wieso man gegen China sein sollte.
                  Der Artikel an sich klingt auch widersprüchlich.
                  Wenn man fürs Klima ist, muss man nicht zwangsläufig gegen China sein.
                  Schließlich betrifft Klima die ganze Welt und nicht nur China.
            \item \interview{P9} \flqq 30.000 Euro geschenkt für einen Tesla\frqq{} klingt doch gut.
            \item \interview{IV} Kannst du das noch näher erläutern?
            \item \interview{P9} An sich finde ich ein elektrisches Auto nicht schlecht.
                  Wenn man dann dafür noch Geld bekommt, ist das gut.
                  Das einzige Problem ist: Woher nimmt man so viel Strom?
                  Bei dem Artikel würde es mich jetzt aber auch interessieren wie viel das Auto insgesamt kostet.
            \item \interview{P9} \flqq Willkommen in der grünen Flammenhölle\frqq{} klingt kontrovers.
                  Den Artikel würde ich klicken.
                  Ich will wissen wie die Grünen handeln.
            \item \interview{P9} \flqq Diess kontert Reitzle\frqq{}, weil der Artikel vermutlich mehrere Meinungen zu dem Thema zeigt.
            \item \interview{P9} Dann \flqq Ein gelber Zettel zeigt, was Kunden mit E-Autos sparen würden\frqq{}.
            \item \interview{IV} Und weswegen würdest du den Artikel klicken?
            \item \interview{P9} Das finanzielle interessiert mich stark.
                  Bei dem einen Artikel kriege ich 30.000 Euro geschenkt, aber bei diesem Artikel könnte ich den Vergleich zu meinem aktuellen Auto ziehen.
            \item {----------------} [Aufgaben - Listenansicht] {----------------}
            \item \interview{IV} Finde und klicke den Artikel \flqq Strom, Trassen, Verteilernetze\frqq{}.
            \item \interview{P9} Hier.
            \item \interview{IV} Finde und klicke den Artikel \flqq Autobauer als Software-Riesen\frqq{}.
            \item \interview{P9} \flqq Autobauer als Software-Riesen\frqq{} ist hier.
            \item \interview{IV} Welcher Artikel ist am ähnlichsten zu \flqq Autobauer als Software-Riesen\frqq{}?
            \item \interview{P9} \flqq Stromrationierung\frqq{} klingt ähnlich.
            \item \interview{IV} Welcher Artikel unterscheidet sich am meisten von \flqq Autobauer als Software-Riesen\frqq{}?
            \item \interview{P9} \flqq Willkommen in der grünen Flammenhölle\frqq{}. Das scheint mir weit weg vom Thema zu sein.
            \item {-----------------} [QUESI - Listenansicht] {-----------------}
            \item {--------------} [AttrakDiff2 - Listenansicht] {--------------}
            \item {---------------------} [FBA - Tutorial] {---------------------}
            \item \interview{P9} Schwein.
            \item \interview{IV} Das Schwein ist leider ein Säugetier.
            \item \interview{P9} Pferd.
            \item \interview{IV} Das Pferd ist leider auch ein Säugetier. Es geht darum was kein Säugetier ist.
            \item \interview{P9} Achso, kein Säugetier. Ich habe das falsch gelesen.
            \item \interview{P9} Wieso ist Rindfleisch bei tierische Nahrungsmittel?
                  Das Tier benötigt ja pflanzliche Nahrungsmittel, um Rindfleisch zu produzieren.
            \item \interview{IV} Das ist richtig und ich verstehe wie du das meinst, aber hier geht es eher um die Kategorisierung.
                  Also eher um das Endprodukt, welches konsumiert wird.
                  Beim Rindfleischburger konsumieren wir sowohl das Fleisch als tierisches Nahrungsmittel als auch das Brötchen als pflanzliches Nahrungsmittel.
            \item \interview{P9} Okay. Ich habe das jetzt verstanden.
            \item \interview{P9} Dann würde ich sagen, dass bei pflanzliche Nahrungsmittel Apfel, Weizen, Hafer und Rindfleischburger dazu gehört.
            \item \interview{IV} Das wären dann wie viele Gegenstände?
            \item \interview{P9} Vier. Und bei Nahrungsmittel sind es dann alle.
            \item \interview{IV} Und das wären wie viele?
            \item \interview{P9} Fünf.
            \item \interview{P9} Nahrungsmittel, pflanzlich und Getreide.
            \item \interview{P9} Das mit den Zahlen verstehe ich gar nicht.
                  Kannst du mir das vielleicht nochmal leichter erklären?
            \item \interview{IV} Zusammengefasst ist bei dieser Ansicht wichtig zu verstehen, dass je höher die Zahl ist, umso mehr unterschiedliche Kategorien werden für die Artikel verwendet.
                  Also in diesem Fall sind die Artikel hier unten in der Kategorie Industrie, welche wir angeklickt haben plus eine weitere Kategorie.
                  In diesem Fall Staat.
                  Ist das etwas verständlicher geworden?
            \item \interview{P9} Ja, etwas.
                  Ich habe das immer noch nicht ganz verstanden, aber das ist okay.
            \item {--------------------------} [FBA] {--------------------------}
            \item \interview{P9} Dann klicke ich hier mal auf die Acht.
                  Also gehören diese Artikel jetzt zu Staat, Markt und Industrie. Richtig?
            \item \interview{IV} Ja, genau.
            \item \interview{P9} Dann klicke ich jetzt auf \flqq Auch ohne Berlin-Werk: Schon heute verdient die dt. Autoindustrie\frqq{}.
            \item \interview{P9} Jetzt auf die Zehn.
                  Dann sehe ich hier nur Markt und Industrie.
                  \flqq Lohnt sich E-Auto bei den Strompreisen noch?\frqq{} klingt interessant, weil ich wie gesagt auch Interesse daran habe wie die Preise sind.
            \item \interview{P9} Hier vorne die Sieben interessiert mich noch.
                  \flqq Deutlich sauberer als gedacht\frqq{}.
            \item \interview{IV} Und weswegen hat dich der Artikel interessiert?
            \item \interview{P9} Einfach, weil ich jetzt auch von Grün etwas sehen will.
            \item \interview{P9} Dann klicke ich jetzt hier oben auf Staat und dann auf \flqq Rohstoffe für E-Auto Akku\frqq{}.
            \item \interview{IV} Weswegen genau dieser Artikel?
            \item \interview{P9} Klingt interessant und ich kann herausfinden, welche Rohstoffe verwendet werden.
            \item \interview{IV} Jetzt vielleicht, da du diesen Artikel vor dir hast.
                  Kannst du die Kategorisierung für diesen Artikel bewerten?
                  Also wie findest du diese Art von Kategorisierung?
            \item \interview{P9} Ich finde die gut.
                  Schließlich hilft das dabei etwas mehr zu verstehen, wieso ein Artikel geschrieben wurde oder aus welcher Richtung er kommt.
            \item \interview{P9} Dann, weil mich der Stromverbrauch interessiert würde ich noch \flqq Stromverbrauch: So viel verbraucht ein E-Auto wirklich\frqq{} lesen.
            \item {---------------------} [Aufgaben - FBA] {---------------------}
            \item \interview{IV} Finde und klicke den Artikel \flqq Harter Schlag für Hersteller Plug-in Prämie fällt weg\frqq{}.
            \item \interview{P9} In Grün würde ich den Artikel vermuten.
            \item \interview{IV} Und dann jetzt noch mal den Artikel \flqq Deutlich sauberer als gedacht\frqq{}.
            \item \interview{P9} Den würde ich auch in Grün vermuten.
            \item \interview{IV} Welcher Artikel ist am ähnlichsten zu \flqq Deutlich sauberer als gedacht\frqq{}?
            \item \interview{P9} \flqq Neue Elektro-Studie unter der Lupe\frqq{}. Beide Artikel klingen nach Forschung.
            \item \interview{IV} Welcher Artikel unterscheidet sich am meisten von \flqq Deutlich sauberer als gedacht\frqq{}?
            \item \interview{P9} Ich würde vermuten bei Markt. \flqq Lohnt sich E-Auto bei den Strompreisen noch?\frqq{}.
            \item {----------------------} [QUESI - FBA] {----------------------}
            \item {-------------------} [AttrakDiff2 - FBA] {-------------------}
            \item {-----------------------} [Fragebogen] {----------------------}
            \item \interview{IV} Welche Darstellungsformen für Online-News-Artikel sind dir bekannt?
            \item \interview{P9} Ich kenne keine anderen.
            \item \interview{IV} Wie häufig besuchst du News-Webseiten?
            \item \interview{P9} Dreimal die Woche.
            \item \interview{IV} Inwieweit hast du bereits Vorerfahrung zum Thema E-Mobilität?
            \item \interview{P9} Ich habe nur etwas davon im Fernsehen gesehen.
            \item \interview{IV} Könntest du dir vorstellen einer der beiden gezeigten Darstellungsformen persönlich zu nutzen?
            \item \interview{P9} Ja, beide. Die Liste kenne ich ja bereits und diese andere Übersicht, weil die etwas Neues ist.
                  Die ist auch generell übersichtlicher, weil die alle Artikel gut anzeigt.
                  Ich habe aber auch nicht alles verstanden, wie das 100\%-ig funktioniert.
            \item \interview{IV} Hast du bereits mit Graphen gearbeitet?
            \item \interview{P9} Nein.
            \item \interview{IV} Denkst du, dass die dir gezeigte neue Darstellungsform ein diverses Bild der Elektromobilität zeigt?
            \item \interview{P9} Wenn man das verstanden hat, dann denke ich schon.
                  Wie gesagt ist das im Vergleich zur Liste übersichtlicher.
            \item \interview{IV} Wie beeinflusst das Aussehen einer Webseite Deine Entscheidung, welche Informationen du liest und wie lange du auf der Seite bleibst?
            \item \interview{P9} Etwas. Ich denke, dass übersichtliche Seiten mich dazu bringen länger zu bleiben.
                  Wenn ich etwas nicht verstehe oder es zu kompliziert ist dann verlasse ich die Seite meistens.
            \item {------------------} [Demografische Daten] {------------------}
            \item \interview{P1} 73, männlich, Realschule, Rentner
      \end{itemize}}
\nolinenumbers
