\section{Transkript - Person 5}
\sloppy
\texttt{\begin{itemize}[]
        \setlength\itemsep{0.02em}
        \linenumbers
        \item {-------------------------} [Intro] {-------------------------}
        \item {---------------------} [Listenansicht] {---------------------}
        \item \interview{P5} Hier fehlt die Werbung, aber ansonsten habe ich glaube ich die Aufgabenstellung verstanden.
        \item \interview{P5} Prinzipiell sind hier schon ein paar Artikel dabei, welche sich sehr nach Clickbait anhören.
              Da würde ich aus Prinzip nicht drauf drücken.
              Zum Beispiel \flqq Jetzt kommen die Wunderakkus\frqq{}.
              Dieser Artikel kommt immer wieder und dann kommen die Wunderakkus vielleicht in 10 Jahren oder so.
        \item \interview{P5} Dann klicke ich mal \flqq Autobauer als Software-Riesen\frqq{}.
              Der Artikel wird nämlich vermutlich über Tesla handeln.
              Ich denke, dass dieser Artikel allgemein auch interessant sein könnte, weil ja auch VW und co. am überlegen sind, Kompetenz in der Software zu haben.
        \item \interview{P5} \flqq Elektro-Zwang?\frqq{} jetzt eher nicht, also klingt zwar interessant, aber auch nach Clicbait.
              Aus diesem Grund würde ich \flqq Fürs Klima und gegen China\frqq{} klicken.
              Das klingt mir ganz allgemein gehalten und interessant.
        \item \interview{P5} Das hier klingt komisch, aber mich würde beim Artikel \flqq 30.000 Euro geschenkt für einen Tesla\frqq{} interessieren, ob das nur für Tesla ist oder nicht.
        \item \interview{IV} Und dein Hauptgrund darauf zu klicken wäre jetzt nur die Neugier, ob das nur auf Tesla zutrifft?
        \item \interview{P5} Ja, prinzipiell schon.
              Aber das Geld wird es ja vermutlich nicht nur für einen Tesla geben.
              Dann würde ich vermuten, dass in dem Artikel beschrieben sein wird wie man das Geld bekommt.
              Und als Interessierter würde ich natürlich auch am Thema Finanzierung für mein Elektroauto interessiert sein.
        \item \interview{P5} Ich würde wahrscheinlich, das mit dem \flqq Ein gelber Zettel zeigt, was Kunden mit E-Autos sparen würden\frqq{} als Nächstes wählen.
              Einfach, um mich noch etwas mehr mit der Finanzierung zu beschäftigen.
        \item \interview{P5} Und dann noch \flqq Herstellungskosten\frqq{} gecheckt, weil es mich ja auch interessiert wie teuer Elektroautos in der Herstellung sind.
              Also speziell zum Beispiel der Akku.
        \item {----------------} [Aufgaben - Listenansicht] {----------------}
        \item \interview{IV} Finde und klicke den Artikel \flqq Strom, Trassen, Verteilernetze\frqq{}
        \item \interview{P5} Hab ich.
        \item \interview{IV} Finde und klicke den Artikel \flqq Autobauer als Software-Riesen\frqq{}
        \item \interview{P5} Und hab ich.
        \item \interview{IV} Welcher Artikel ist am ähnlichsten zu \flqq Autobauer als Software-Riesen\frqq{}
        \item \interview{P5} Vielleicht \flqq Herstellungskosten gecheckt\frqq{}.
              Aber auch \flqq Diess kontert Reitzle\frqq{}.
              Dann würde ich mich vermutlich für \flqq Diess kontert Reitzle\frqq{} entscheiden.
              Diess hat ja, sofern ich weiß, irgendwelche Entscheidungen zu Software getroffen und das könnte zum vorherigem Artikel passen.
        \item \interview{IV} Welcher Artikel unterscheidet sich am meisten von \flqq Autobauer als Software-Riesen\frqq{}.
        \item \interview{P5} \flqq Willkommen in der grünen Flammenhölle\frqq{}?
              Wegen des Titels und dem Wort Flammenhölle hätte ich es nicht einmal erwartet, dass es sich um das Thema E-Mobilität handelt.
        \item {-----------------} [QUESI - Listenansicht] {-----------------}
        \item {--------------} [AttrakDiff2 - Listenansicht] {--------------}
        \item {---------------------} [FBA - Tutorial] {---------------------}
        \item {--------------------------} [FBA] {--------------------------}
        \item {---------------------} [Aufgaben - FBA] {---------------------}
        \item \interview{IV} Finde und klicke den Artikel \flqq Harter Schlag für Hersteller Plugin Prämie fällt weg\frqq{}.
        \item \interview{IV} Und dann jetzt noch mal den Artikel \flqq Deutlich sauberer als gedacht\frqq{}.
        \item \interview{IV} Welcher Artikel ist am ähnlichsten zu \flqq Deutlich sauberer als gedacht?\frqq{}?
        \item \interview{IV} Welcher Artikel unterscheidet sich am meisten von \flqq Deutlich sauberer als gedacht?\frqq{}?
        \item {----------------------} [QUESI - FBA] {----------------------}
        \item {-------------------} [AttrakDiff2 - FBA] {-------------------}
        \item {-----------------------} [Fragebogen] {----------------------}
        \item \interview{IV} Welche Darstellungsformen für Online-News-Artikel sind Dir bekannt?
        \item \interview{IV} Wie häufig besuchst Du News-Webseiten?
        \item \interview{IV} Inwieweit hast du bereits Vorerfahrung zum Thema E-Mobilität?
        \item \interview{IV} Könntest Du Dir vorstellen einer der beiden gezeigten Darstellungsformen persönlich zu nutzen?
        \item \interview{IV} Hast Du bereits mit Graphen gearbeitet?
        \item \interview{IV} Denkst Du, dass die Dir gezeigte neue Darstellungsform ein diverses Bild der Elektromobilität zeigt?
        \item \interview{IV} Wie beeinflusst das Aussehen einer Webseite Deine Entscheidung, welche Informationen Du liest und wie lange Du auf der Seite bleibst?
        \item {------------------} [Demografische Daten] {------------------}
    \end{itemize}}
\nolinenumbers
