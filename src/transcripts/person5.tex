\section{Transkript - Person 5}
\sloppy
\texttt{\begin{itemize}[]
            \setlength\itemsep{0.02em}
            \linenumbers
            \item {-------------------------} [Intro] {-------------------------}
            \item {---------------------} [Listenansicht] {---------------------}
            \item \interview{P5} Hier fehlt die Werbung, aber ansonsten habe ich glaube ich die Aufgabenstellung verstanden.
            \item \interview{P5} Prinzipiell sind hier schon ein paar Artikel dabei, welche sich sehr nach Clickbait anhören.
                  Da würde ich aus Prinzip nicht drauf drücken.
                  Zum Beispiel \flqq Jetzt kommen die Wunderakkus\frqq{}.
                  Dieser Artikel kommt immer wieder und dann kommen die Wunderakkus vielleicht in 10 Jahren oder so.
            \item \interview{P5} Dann klicke ich mal \flqq Autobauer als Software-Riesen\frqq{}.
                  Der Artikel wird nämlich vermutlich über Tesla handeln.
                  Ich denke, dass dieser Artikel allgemein auch interessant sein könnte, weil ja auch VW und co. am überlegen sind, Kompetenz in der Software zu haben.
            \item \interview{P5} \flqq Elektro-Zwang?\frqq{} jetzt eher nicht, also klingt zwar interessant, aber auch nach Clicbait.
                  Aus diesem Grund würde ich \flqq Fürs Klima und gegen China\frqq{} klicken.
                  Das klingt mir ganz allgemein gehalten und interessant.
            \item \interview{P5} Das hier klingt komisch, aber mich würde beim Artikel \flqq 30.000 Euro geschenkt für einen Tesla\frqq{} interessieren, ob das nur für Tesla ist oder nicht.
            \item \interview{IV} Und dein Hauptgrund darauf zu klicken wäre jetzt nur die Neugier, ob das nur auf Tesla zutrifft?
            \item \interview{P5} Ja, prinzipiell schon.
                  Aber das Geld wird es ja vermutlich nicht nur für einen Tesla geben.
                  Dann würde ich vermuten, dass in dem Artikel beschrieben sein wird wie man das Geld bekommt.
                  Und als Interessierter würde ich natürlich auch am Thema Finanzierung für mein Elektroauto interessiert sein.
            \item \interview{P5} Ich würde wahrscheinlich, das mit dem \flqq Ein gelber Zettel zeigt, was Kunden mit E-Autos sparen würden\frqq{} als Nächstes wählen.
                  Einfach, um mich noch etwas mehr mit der Finanzierung zu beschäftigen.
            \item \interview{P5} Und dann noch \flqq Herstellungskosten\frqq{} gecheckt, weil es mich ja auch interessiert wie teuer Elektroautos in der Herstellung sind.
                  Also speziell zum Beispiel der Akku.
            \item {----------------} [Aufgaben - Listenansicht] {----------------}
            \item \interview{IV} Finde und klicke den Artikel \flqq Strom, Trassen, Verteilernetze\frqq{}.
            \item \interview{P5} Hab ich.
            \item \interview{IV} Finde und klicke den Artikel \flqq Autobauer als Software-Riesen\frqq{}.
            \item \interview{P5} Und hab ich.
            \item \interview{IV} Welcher Artikel ist am ähnlichsten zu \flqq Autobauer als Software-Riesen\frqq{}?
            \item \interview{P5} Vielleicht \flqq Herstellungskosten gecheckt\frqq{}.
                  Aber auch \flqq Diess kontert Reitzle\frqq{}.
                  Dann würde ich mich vermutlich für \flqq Diess kontert Reitzle\frqq{} entscheiden.
                  Diess hat ja, sofern ich weiß, irgendwelche Entscheidungen zu Software getroffen und das könnte zum vorherigem Artikel passen.
            \item \interview{IV} Welcher Artikel unterscheidet sich am meisten von \flqq Autobauer als Software-Riesen\frqq{}?
            \item \interview{P5} \flqq Willkommen in der grünen Flammenhölle\frqq{}?
                  Wegen des Titels und dem Wort Flammenhölle hätte ich es nicht einmal erwartet, dass es sich um das Thema E-Mobilität handelt.
            \item {-----------------} [QUESI - Listenansicht] {-----------------}
            \item {--------------} [AttrakDiff2 - Listenansicht] {--------------}
            \item {---------------------} [FBA - Tutorial] {---------------------}
            \item \interview{P5} Jetzt ist also mein Allgemeinwissen gefragt.
            \item \interview{P5} Bei pflanzliche Nahrungsmittel wären, das eins plus die drei plus den Apfel wären das fünf.
                  Und oben wären das sieben.
                  Obwohl, warte mal.
                  Dann wäre der Rindfleischburger doppelt gezählt.
                  Und ich glaube, dass das mit den fünf auch nicht ganz passt.
            \item \interview{P5} Ich zähle die mal von Hand durch. Eins, zwei, drei, vier. Also vier bei pflanzliche Nahrungsmittel.
            \item \interview{IV} Und oben bei Nahrungsmittel? Bleibst du da bei sieben?
            \item \interview{P5} Das sollten die vier plus eins, also fünf sein.
            \item \interview{IV} Alles klar. Dann musst du mir nur noch verraten, wo Hafer sich einordnen lässt.
            \item \interview{P5} Nahrungsmittel, pflanzliche Nahrungsmittel und dann Getreide.
            \item {--------------------------} [FBA] {--------------------------}
            \item \interview{P5} So richtig weiß ich nicht, was ich machen soll, weil ich mich dann ja bewusst entscheiden muss.
                  Also, will ich was zum Beispiel Richtung Staat und Industrie?
            \item \interview{IV} Das bleibt dir überlassen für welche Kategorien du dich entscheidest.
            \item \interview{P5} Aber ich meine als Konsument prinzipiell.
            \item \interview{IV} Möglicherweise habe ich dich dann nicht richtig verstanden. Kannst du das vielleicht etwas näher erläutern?
            \item \interview{P5} Ich bin generell nicht entscheidungsfreudig und bin eher Fan davon, wenn man mir Artikel vor die Füße wirft.
                  Und wenn ich etwas direkt suche, dann würde ich einfach Google benutzen.
            \item \interview{IV} Also willst du damit aussagen, dass du Vorschlagsysteme bevorzugst, weil du dann nicht aktiv auf die Suche gehen musst?
            \item \interview{P5} Ja, genau.
                  Umso besser der dahinterliegende Algorithmus ist, umso besser sind die Vorschläge.
                  Natürlich bin ich kein Fan davon, wenn mir nur Dreck vorgeschlagen wird.
                  Dann habe ich auch keinen Bock darauf.
                  Aber wenn es gut ist, also Dinge, welche zu mir passen würde ich so ein System bevorzugen.
            \item \interview{P5} Wenn wir jetzt hier dieses System vorliegen haben, dann würde ich sagen, dass ich mich beim Artikel suchen und bilden nicht für Kategorien entscheiden würde.
            \item \interview{P5} Aber ich glaube, ich würde in diesem System \flqq Neue Elektro-Studie unter der Lupe\frqq{} als Erstes lesen.
                  Prinzipiell finde ich es interessant, aber leider gibt es auch viele Studien, welche Schwachsinn erzählen.
                  Aber, wenn diese eine Tatsache näher erforschen finde ich das interessant.
                  Nur ob alles auch stimmt oder Dinge bei Rechnungen ignoriert werden, ist ja immer die Frage.
                  Teils ja auch mit Absicht.
            \item \interview{IV} Könntest du noch etwas zu der Kategorisierung selbst beim Artikel sagen? Also vielleicht eine Art Bewertung dafür geben?
            \item \interview{P5} Prinzipiell finde ich das nicht schlecht, weil wenn ich das jetzt richtig verstanden habe, dann beleuchtet dieser Artikel ja in mehreren Kategorien beide Seiten.
                  Vorhin war aber im Tutorial noch Inhalt, welcher im Text hervorgehoben wurde. Gibt es einen Grund dafür, wieso das nicht mehr da ist?
            \item \interview{IV} Das ist aktuell noch kein Feature, welches eingebaut ist.
                  Das diente im Tutorial nur dafür, dass man ein Gefühl dafür bekommt, wie die Kategorisierung funktioniert.
                  Würdest du dir wünschen, dass das Feature auch im System implementiert wird?
            \item \interview{P5} Prinzipiell würde das, glaube ich, eher beim Lesen ablenken.
                  Vielleicht eher so, dass man auf eine Kategorie klickt und dann wird der Text hervorgehoben.
            \item \interview{IV} Das ist eine gute Idee.
            \item \interview{P5} Den Artikel \flqq Stromverbrauch: So viel verbraucht ein E-Auto wirklich\frqq{} würde ich lesen, weil ähnlich wie bei der Liste mich die finanziellen Aspekte interessieren.
                  Einfach, um die Kosten abzuschätzen, welche auf einen zukommen.
                  Vielleicht auch, um zu sehen, wie energieeffizient das Auto ist.
                  Also speziell im Vergleich zu einem Verbrenner.
            \item \interview{P5} Und scheinbar ist ein E-Auto das von der Kategorie her.
                  Also, wenn ich mich richtig erinnere stand nämlich Industrie für Effizienz und hier ist die Kategorie grün.
                  Das nimmt allerdings auch etwas Inhalt vorweg.
            \item \interview{IV} Findest du das positiv oder negativ?
            \item \interview{P5} Eher positiv.
                  Hat etwas von einer Zusammenfassung, aber, wenn man den genauen Inhalt wissen will, dann kann man den Artikel ja lesen.
                  Also ähnlich wie kleine Header, welche manchmal einen Artikel beschreiben.
            \item \interview{P5} Dann würde ich noch \flqq autonom. Fahren Erfolg oder Flop?\frqq{} klicken, weil von Tesla wird ja viel berichtet, aber vielleicht beleuchtet der Artikel auch wie es bei den anderen läuft.
            \item \interview{P5} Dass hier die Knoten ausgegraut werden ist gewollt?
                  Also, dass keine Überschrift mehr über dem Knoten steht?
            \item \interview{IV} Aktuell ist das gewollt. Hättest du dazu denn eine Idee oder Feedback wie es besser sein könnte?
            \item \interview{P5} Ja. Also mir fehlt es ein bisschen, wenn ich jetzt auf Industrie klicke, sehe ich zwar die Verbindung zu Grün, aber ich muss das im Kopf behalten.
            \item \interview{IV} Ich verstehe. Das ist definitiv auch eine gute Idee, welche ich noch umsetzen werde. Danke für dein Feedback.
            \item \interview{P5} Das gelbe ist bei Grün dann vermutlich anteilmäßig, korrekt?
            \item \interview{IV} Genau.
            \item \interview{P5} Alles klar. Das hätte ich so jetzt auch intuitiv gedacht.
            \item \interview{P5} Ich klicke jetzt auf \flqq Rohstoffe für E-Auto Akku\frqq{}, weil der Titel nicht wertend klingt.
                  Aber meine Interpretation aus der Kategorie Staat mit roter Farbe wäre jetzt, dass es scheinbar Probleme mit den Regulierungen gibt.
            \item \interview{P5} Meine Navigation kommt dem Vorschlagsalgorithmus gleich.
                  In dem Sinne, dass ich mir sozusagen einen Vorschlag aussuche.
                  Also sozusagen: Schlag mir etwas von der Kategorie Staat vor.
            \item \interview{IV} Dunavigierst generell hauptsächlich über die Kategorien. Hat das einen bestimmten Grund?
            \item \interview{P5} Ja. Ich glaube das liegt daran, dass diese beschriftet sind und mehr Inhalt aufgelistet wird.
            \item \interview{IV} Würde es dir helfen, wenn die zusammengesetzten Knoten auch beschriftet wären?
            \item \interview{P5} Nein. Ich glaube, dass das eher verwirren würde und zu viel Text wäre, wenn jetzt überall noch die oberen Kategorien stehen würden.
            \item \interview{IV} Würde dir vielleicht eine temporäre Beschriftung helfen? Also, wenn du über einen zusammengesetzten Knoten hoverst, dass dann wird die Beschriftung angezeigt?
            \item \interview{P5} Ja, ich glaube, das wäre schon besser.
            \item \interview{P5} Dann noch der Artikel \flqq Auch ohne Berlin-Werk: Schon heute verdient dt. Autoindustrie\frqq{}.
                  Einfach, um auch die deutsche Wirtschaft etwas anzuschauen.
                  Ob die Probleme bekommt oder nicht.
            \item {---------------------} [Aufgaben - FBA] {---------------------}
            \item \interview{IV} Finde und klicke den Artikel \flqq Harter Schlag für Hersteller Plug-in Prämie fällt weg\frqq{}.
            \item \interview{P5} Den habe ich vorhin schon gesehen.
                  Dann würde ich den Knoten Staat anklicken und das darüber machen.
            \item \interview{IV} Und dann jetzt noch mal den Artikel \flqq Deutlich sauberer als gedacht\frqq{}.
            \item \interview{P5} Den würde ich bei Grün vermuten.
            \item \interview{IV} Welcher Artikel ist am ähnlichsten zu \flqq Deutlich sauberer als gedacht\frqq{}?
            \item \interview{P5} Das wäre natürlich einfach gewesen, wenn im selben Knoten einer drinnen wäre.
                  Na dann was darunter liegt noch. Also das wird dann ja dieselben Kategorien aufweisen.
                  Man könnte auch weiter nach oben gehen zu Industrie, aber würde dann eine Kategorie verlieren.
                  Deswegen gehe ich mal weiter nach unten.
                  Also von mir aus einfach \flqq Rohstoffe für E-Auto Akku\frqq{}.
            \item \interview{IV} Welcher Artikel unterscheidet sich am meisten von \flqq Deutlich sauberer als gedacht\frqq{}?
            \item \interview{P5} Hier würde ich in die komplett andere Richtung gehen.
                  Dann einfach den Artikel \flqq autonom. Fahren Erfolg oder Flop?\frqq{}
            \item {----------------------} [QUESI - FBA] {----------------------}
            \item {-------------------} [AttrakDiff2 - FBA] {-------------------}
            \item {-----------------------} [Fragebogen] {----------------------}
            \item \interview{IV} Welche Darstellungsformen für Online-News-Artikel sind dir bekannt?
            \item \interview{P5} Generell Kategorisierung, aber dann halt auch in Listen. Also zum Beispiel Sport als Kategorie.
                  Ich hatte auch mal eine News-App, welche alles anders machen wollte, aber irgendwie hatte das nicht ganz funktioniert und leider erinnere ich mich nicht mehr daran wie das genau war.
            \item \interview{IV} Wie häufig besuchst du News-Webseiten?
            \item \interview{P5} Einmal bis zweimal pro Tag.
            \item \interview{IV} Inwieweit hast du bereits Vorerfahrung zum Thema E-Mobilität?
            \item \interview{P5} Recht viel, würde ich sagen.
            \item \interview{IV} Könntest du dir vorstellen einer der beiden gezeigten Darstellungsformen persönlich zu nutzen?
            \item \interview{P5} Ja, beide. Aber ich würde vermutlich die Liste aufgrund von Bequemlichkeit bevorzugen.
            \item \interview{IV} Hast du bereits mit Graphen gearbeitet?
            \item \interview{P5} Ja.
            \item \interview{IV} Denkst du, dass die dir gezeigte neue Darstellungsform ein diverses Bild der Elektromobilität zeigt?
            \item \interview{P5} Es war jetzt zwar viel grün, also positiv und ich hätte es jetzt so verstanden, dass es mir eher nur ein positives Bild vermittelt.
            \item \interview{IV} Und, wenn wir das jetzt nur auf die Darstellungsform beschränken und mehr Artikel enthalten wären, welche auch negativ bezüglich einer Kategorie berichten?
                  Vielleicht auch im Vergleich zur Darstellungsform mit Listen?
            \item \interview{P5} Ja, dann schon. Ich glaube, dass der Graph ein diverses Bild vermittelt.
                  Bei der Liste hat man schließlich keine Ahnung wie Artikel in Zusammenhang stehen oder welcher Kategorie diese angehören.
            \item \interview{IV} Wie beeinflusst das Aussehen einer Webseite Deine Entscheidung, welche Informationen du liest und wie lange du auf der Seite bleibst?
            \item \interview{P5} Sehr. Wenn viel Werbung auf der Seite angezeigt wird, bin ich sofort raus.
            \item {------------------} [Demografische Daten] {------------------}
            \item \interview{P5} 31, männlich, Bachelor, Informatiker
      \end{itemize}}
\nolinenumbers
