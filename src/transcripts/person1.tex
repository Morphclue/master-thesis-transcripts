\newcommand{\interview}[1]{[#1]}
\section{Transkript - Person 1}
\sloppy
\texttt{\begin{itemize}[]
    \setlength\itemsep{0.02em}
    \linenumbers
    \item {-------------------------} [Intro] {-------------------------}
    \item {---------------------} [Listenansicht] {---------------------}
    \item \interview{P1} Also das Erste, was ich versuche, wenn ich auf eine Seite gehe, ist zu scrollen. 
    Auf der Seite kann man jetzt nicht scrollen, deswegen fällt das weg. 
    Aber wenn ich mir dann die Artikel auf der rechten Seite angucke, dann würde ich die zuerst anklicken, die mir direkt ins Auge springen, wenn es um Elektro geht.
    \item \interview{P1} Also beispielsweise \flqq fürs Klima, gegen China\frqq{}.
    China hat halt nur indirekt was damit zu tun.
    Deswegen wenn ich mir jetzt ein Elektroauto kaufen will, dann würde ich halt eher so Sachen anklicken wie \flqq Elektrozwang\frqq{} oder was ich gesehen hab \flqq 30.000€ geschenkt für einen Tesla\frqq{}.
    Das wären so die Sachen, die ich zuerst anklicke.
    Deswegen gehe ich jetzt einfach mal direkt auf \flqq Elektrozwang\frqq{}.
    Vor allem, weil es natürlich ein interessantes Thema ist, was mich ja dann auch betreffen könnte, wenn ich mich damit auseinandersetzen.
    \item \interview{P1} Genau, nachdem ich mir den durchgelesen habe, gucke ich mir einfach die nächsten Vorschläge an.
    Also wie gesagt, ich würde dann halt so weitergehen und die zuerst anklicken, die mir direkt ins Auge springen.
    Zum Beispiel Tesla weiß ich halt, dass das was mit E-Mobilität zu tun hat.
    Und wenn ich Geld geschenkt kriege, dann interessiert mich das natürlich auch.
    Also würde ich das halt anklicken.
    \item \interview{P1} Also ich würd mir jetzt spontan nicht mehr Gedanken darüber machen. 
    Also dadurch, dass ich jetzt kein direktes Thema habe, was mich interessiert, sondern ich mich halt für die E-Mobilität allgemein interessiere, würde ich halt einfach alles konsumieren was ich in die Finger bekomme.
    Und deswegen würde ich jetzt nicht nach irgendwelchen speziellen Themen Ausschau halten, sondern einfach alles, was damit zu tun hat.
    \item \interview{P1} Und wenn ich dann mit den Sachen, die direkt auf die E-Autos hindeuten, durch bin, also beispielsweise hier sehe ich halt, hier steht noch E-Auto oder irgendwo stand gerade was mit Auto Bloggerin.
    Nachdem ich die durch habe, würde ich dann halt in Richtung Strom gehen.
    Das wäre so mein Vorgehen, dass ich halt zuerst das, was wirklich direkt sich ums Auto dreht, anklicke und dann im Nachhinein halt alles, was mehr oder weniger indirekt damit zu tun hat.
    \item \interview{IV} Deine Reihenfolge bislang war jetzt \flqq Elektrozwang\frqq{}, \flqq 30.000 Euro für einen Tesla\frqq{}, \flqq Ein gelber Zettel zeigt, was Kunden mit E-Autos sparen würden\frqq{} und \flqq Auto-Bloggerin nimmt Tesla-Modell auseinander\frqq{}, korrekt?
    \item \interview{P1} Genau.
    Also ich habe es mir halt von oben nach unten die Liste durchgelesen und habe dann halt einfach die ersten angeklickt, die mir direkt ins Auge gesprungen sind.
    Also, dass ich die Reihenfolge, in der ich die angeklickt habe, hatte jetzt tatsächlich was damit zu tun, in welcher Reihenfolge die auch aufgeführt wurden.
    \item \interview{IV} Alles klar. Dann informiere Dich noch zu einem weiteren Artikel und dann könnten wir auch erstmal stoppen.
    \item \interview{P1} Ich denke mal, die, die ich jetzt direkt ins Auge gesprungen sind, habe ich jetzt schon angeklickt.
    Wie gesagt, dann würde ich jetzt halt weitermachen mit so Sachen wie Strom, also beispielsweise \flqq Stromrationierung\frqq{}. Das ist jetzt so ein gutes Stichwort, was mir direkt ins Auge springt.
    \item {----------------} [Aufgaben - Listenansicht] {----------------}
    \item \interview{IV} Finde und klicke den Artikel \flqq Strom, Trassen, Verteilernetze\frqq{}.
    \item \interview{P1} Check.
    \item \interview{IV} Finde und klicke den Artikel \flqq Autobauer als Software-Riesen\frqq{}.
    \item \interview{P1} Hier.
    \item \interview{IV} Welcher Artikel ist am ähnlichsten zu \flqq Autobauer als Software-Riesen\frqq{}?
    \item \interview{P1} Das ist eine gute Frage. 
    Also bei Software-Riesen: Das hört man natürlich immer im Zusammenhang mit Tesla.
    Ich habe jetzt irgendwas im Zusammenhang mit Autobauer und Riese gesucht, aber das sehe ich jetzt hier nicht.
    Das wäre mir nämlich als Erstes in den Sinn gekommen.
    Halt einfach ein großer Autobauer.
    Aber zweite Gedanke wäre dann halt alles, was irgendwie mit Tesla zu tun hat wegen Software wie.
    Vielleicht aus dem Gefühl heraus \flqq Auto-Bloggerin nimmt Tesla-Modell auseinander\frqq{}. 
    Einfach, weil es da wahrscheinlich sehr viel um Software gehen wird und um den Autobauer an sich.
    \item \interview{IV} Welcher Artikel unterscheidet sich am meisten von \flqq Autobauer als Software-Riesen\frqq{}?
    \item \interview{P1} Also ich hätte jetzt so was gesagt wie beispielsweise \flqq fürs Klima und gegen China\frqq{}. 
    Zumindest vom Titel her.
    Es hat jetzt nichts direkt mit Autos zu tun hat, sondern Klima geht ja weit über die E-Mobilität hinaus.
    Von daher würde ich da jetzt erst mal nicht gucken. 
    Vielleicht, weil es sich halt um E-Autos dreht schon. 
    Aber grundsätzlich würde ich halt, wenn es jetzt das Thema Klima ist nicht direkt daran denken, dass es irgendwas mit E-Mobilität zu tun hat.
    \item {-----------------} [QUESI - Listenansicht] {-----------------}
    \item {--------------} [AttrakDiff2 - Listenansicht] {--------------}
    \item {---------------------} [FBA - Tutorial] {---------------------}
    \item \interview{P1} Die Frage ist klar. 
    Also es wird die Kategorie Säugetier gesucht und es wird einfach gefragt, welches Tier können wir auch zuordnen?
    \item \interview{P1} Okay, die Darstellung, die finde ich interessant.
    \item \interview{P1} Ich dachte zuerst, dass hier zwei entgegengesetzte Dinge dargestellt werden sollen. Aber der Kreis, der ergibt jetzt eine Einheit sozusagen.
    \item \interview{P1} Okay, jetzt haben wir den Gegensatz gesucht, also war auch wieder klar.
    \item \interview{P1} Okay, jetzt werden Relationen hergestellt.
    \item \interview{P1} Okay, das muss ich erst mal verarbeiten.
    Also genau wie die hier schon schreiben.
    Ich meine, es ist ja ein Tutorial, das erklärt mir das ja auch gerade zum ersten Mal.
    Deswegen kann ich nicht viel mehr dazu sagen als das, was ich hier selber lese.
    Aber es werden halt Relationen hergestellt und ist jetzt halt mit dem Tier verknüpft, weil das hier quasi in diesem Kreis integriert ist und hat dann halt durch diese Relation eine Verknüpfung zum Säugetier und wurde deswegen hier diese Anzahl auf zwei erhöht.
    Also man kann Verknüpfungen auf unterschiedliche Arten herstellen.
    Eine Kategorie, welche sich unter Säugetiere einordnen lässt.
    Ich denke Fleischfressern. Ah okay, und das wird auch scheinbar so weitergereicht.
    Also oben ist jetzt hat sich jetzt die 2 in eine 3 verwandelt, obwohl es keine direkte Relation war, sondern eine indirekte Relation.
    \item \interview{P1} Okay, das bedeutet, weil wir hier jetzt zwei Tiere quasi zugeordnet haben oder zwei Objekte dieser Kategorie zugeordnet haben, hat sich hier die Zahl auch auf drei erhöht.
    So würde ich das jetzt verstehen.
    Steht jetzt zwar nicht explizit hier, aber wurde ja beschrieben, dass die Kuh hinzugefügt wurde.
    Und das war ja jetzt so, würde ich jetzt so interpretieren, als dass deshalb hier die Zahl sich erhöht hat.
    Ja, oder es kann natürlich auch sein gut, dass dadurch, dass keine Zahl hier doppelt vorkommt, könnte man auch denken, dass das einfach eine aufsteigende Nummerierung ist.
    Jetzt 1, 2, 3, 4, 5 aber gehe ich, beziehungsweise wäre ich jetzt erst mal nicht von ausgegangen, dass das so sein soll, würde ich aber im Hinterkopf behalten.
    Also da würde mir wird es mir leichter fallen, das zu verstehen, wenn man ein Beispiel hätte, wo auch Zahlen doppelt vorkommen würden.
    \item \interview{IV} An dieser Stelle reicht es, wenn Du mir die Ergebnisse mündlich mitteilst. 
    \item \interview{P1} Wie viele Gegenstände besitzt die Kategorie pflanzliche Nahrungsmittel? Da würde ich jetzt gucken pflanzliche Nahrungsmittel ist hier, hat untergeordnet den Apfel und untergeordnet Weizen und Hafer und hat untergeordnet Rindfleischburger.
    Also wären das in dem Fall jetzt vier untergeordnete.
    Deswegen würde ich da jetzt eine vier einfügen.
    \item \interview{P1} Wie viel Gegenstände besitzt die Kategorie Nahrungsmittel ist hier oben verteilt, die ihr von hier links als untergeordnete plus die - Ah okay, hier ist jetzt.
    Also hier kommt jetzt nur eins dazu, weil der Rindfleischburger quasi sonst doppelt gezählt werden würde.
    Deswegen hätte ich hier nur eins hinzugefügt.
    Wäre dann in dem Fall fünf. 
    Genau, und welche Kategorie hat der Gegenstand Hafer? 
    Getreide.
    \item \interview{IV} Alles klar.
    Dann kannst Du jetzt auf fertig drücken.
    \item \interview{P1} Ich glaube, ich habe es verstanden.
    Okay, also ein Artikel kann ja einfach in unterschiedlichen Welten spielen, aber das ist unabhängig von der Kategorie so wie ich das jetzt verstehe.
    Also dieser lilane Teil könnte jetzt eine Welt und einer Kategorie zugeordnet sein, so wie ich es jetzt verstanden habe.
    \item \interview{P1} Also soll jetzt das ganze Ding eine Welt sein?
    Dann war vielleicht meine Vermutung von vorher doch nicht ganz richtig.
    \item \interview{IV} Dann hake ich hier mal kurz ein, um das richtigzustellen.
    Eine Welt kannst Du gleichsetzen mit einer Kategorie.
    Also eine Welt ist eine Form der Kategorisierung.
    \item \interview{P1} Dann würde es sich mir nicht ergeben, wieso wir das überhaupt unterschieden? 
    Aber das kommt ja vielleicht noch später im Tutorial.
    \item \interview{P1} Die Farben sind einleuchtend. 
    Also einfach nur eine Bewertung, eine Gewichtung, sage ich mal oder eine Wertung, ob es positiv, negativ oder neutral ist, würde ich das dann sehen.
    Ja, genau.
    Aus beiden, also sowohl positiv als auch negativ wird dieser Artikel behandelt. 
    Also neutral in dem Fall oder objektiv.
    Sagen wir das mal so.
    \item \interview{IV} Bevor du gleich noch den nächsten Klick machst.
    Ist das verständlich soweit?
    \item \interview{P1} Gut, also ich sehe was da beschrieben ist.
    Also ich verstehe jetzt nicht sofort zu 100\% was gleich passieren wird, aber ich bin auch eher so der Typ, der sich eine Sache erst mal so oberflächlich durchliest, sage ich mal und dann einfach ausprobiert, was passiert.
    Ich schätze mal, dass es da unterschiedliche Typen geben wird, also dass andere Leute da anders mit umgehen würden.
    Aber ich bin jetzt nicht davor davon abgeschreckt, irgendwo drauf zu klicken, selbst wenn ich den Text jetzt nicht zu 100\% verstehe.
    Aber wie gesagt, also zu 100\% verstehe ich jetzt nicht, was passieren wird.
    Aber ich denke mir dann, dass ich das gleich einfach sehen werde.
    \item \interview{P1} Okay, man kann nirgendwo anders draufklicken, da war ich gerade neugierig, ob ich auch andere Sachen machen könnte, als das, was mir gesagt hat.
    \item \interview{P1} Ah, okay, jetzt wird das auch klar mit der Leserichtung nach unten und nach oben.
    Dadurch, dass die Kategorie oben ist, werden quasi die Artikel von oben nach unten aufgeführt.
    Wenn Du einen Artikel klickst, also unten anklicken, dann werden die Kategorien quasi von unten nach oben aufgelistet.
    \item {--------------------------} [FBA] {--------------------------}
    \item \interview{P1} Hier kann man hovern.
    Aber das wurde ja zum Glück auch schon erklärt.
    \item \interview{P1} Gut, also was ich wahrscheinlich zuerst machen würde, ist, dass ich mir halt erst mal einen Überblick über die Kategorien verschaffe.
    Oder die Welten.
    Wie gesagt, da bin ich mir jetzt immer noch nicht 100\%-ig sicher, wieso diese Unterscheidung jetzt überhaupt gemacht wurde.
    Aber da würde ich mir jetzt erst mal einen Überblick verschaffen, was es hier so gibt.
    Ich sehe Staat, Markt, Industrie und Grün und kann dort gucken, wie so die Verteilung ist.
    Also wie die so bewertet sind.
    Und ich würde jetzt erst mal anfangen, mich mit den vorwiegend positiven Punkten zu beschäftigen, damit ich erst mal grundsätzlich einen Einstieg habe, der jetzt nicht - also ich bin ja interessiert, mich mit diesem Thema auseinanderzusetzen.
    Natürlich auch kritisch, aber ich möchte halt erst mal wissen, was denn die Community selbst sozusagen als Vorteile von diesem Thema sieht.
    Und deswegen würde ich halt jetzt erst mal die mich auf die Artikel konzentrieren oder auf die Welten konzentrieren, die halt positiv bewertet sind.
    Wobei natürlich überall ein positiver Anteil drin ist.
    Deswegen gehe ich jetzt einfach mal zuerst auf den Markt klicken, weil der halt durchgehend positiv ist.
    \item \interview{P1} Und dann sehe ich ja schon, dass ich hier die Artikel aufgelistet kriege.
    So wie ich es jetzt verstehe, dem Markt sind jetzt keine Artikel zugeordnet und hier wäre es jetzt ein bisschen schwierig für mich zu verstehen.
    Ach so, das sind ah, okay, das ist mir eben gerade auch noch nicht klar geworden, diese Zahlen, die geben nicht bestimmte Welten an oder bestimmte Zustände an.
    Also ich habe es so verstanden, dass jetzt die Null beispielsweise der Markt ist und die Eins, dann der nachfolgende Zustand.
    Aber was hier wirklich ausgesagt wird ist die Ebene, die angegeben wird.
    Also eine Ebene darunter wären dann die Artikel \flqq Stromverbrauch [...]\frqq{} und so weiter.
    Und \flqq Lohnt sich ein E-Auto [...]\frqq{}.
    Die würden dann so wie ich es jetzt verstehe hier am Staat und an diesem unbenannten, an dieser unbenannten Kategorie hängen.
    Und dann zweite Ebene wären halt hier die beiden, also die mit der acht und der sieben und dann dritte Ebene wäre hier unten die letzte Kategorie.
    \item \interview{P1} Genau, aber die, ich sage mal die Reihenfolge der Artikel sagt mir jetzt nicht wirklich viel beziehungsweise sagt natürlich schon was, aber ich würde da jetzt erst mal nicht drauf achten, weil ich mich ja insgesamt über das Thema informieren will.
    Also ich möchte jetzt nicht ein Überblick in einem bestimmten Bereich haben, sondern ich möchte mich insgesamt mit dem Thema beschäftigen.
    Deswegen ist es für mich jetzt eigentlich erst mal egal, zu welcher Kategorie ein Artikel gehört.
    Und deswegen würde ich jetzt quasi so vorgehen, wie es bei der letzten Ansicht auch war. 
    Ich habe mir jetzt wie gesagt zuerst die positiven Artikel herausgesucht oder die, wo ich davon ausgehe, dass sie größtenteils positiv sind und würde dann halt einfach die anklicken, die mir als Erstes ins Auto - ins Auto, ich meine ins Auge springen.
    \item \interview{P1} Keine Ahnung.
    Einfach den Ersten. 
    \flqq Stromverbrauch: So viel verbraucht ein E-Auto wirklich\frqq{} und dann würde ich mir den halt angucken.
    Okay, und hier sehe ich jetzt auch noch mal genau, zu welchen Kategorien der Artikel gehört, was natürlich cool ist.
    Und dann sehe ich auch okay, jetzt habe ich auch genau so einen erwischt, wie ich eigentlich erwartet habe, der in allen Fällen positiv ist.
    Ich würde mir den durchlesen und dann halt wieder zurück zu der Übersicht gehen und mich so ein bisschen durcharbeiten.
    \item \interview{P1} Vielleicht würde ich einfach mal auf den zweiten Artikel klicken.
    Aber in dem Fall jetzt genauso einer, den ich irgendwie erwartet habe, also halt größtenteils positiv ist.
    Und ich denke, so würde ich jetzt erst mal für so ein paar Artikel vorgehen, bevor ich dann - Also ich würde mir garantiert nicht alle Artikel durchlesen, sondern halt erst mal nur so einen groben Überblick verschaffen.
    Ich würde schauen: Welche positiven Artikel gibt es dann?
    \item \interview{P1} Und dann würde ich aber irgendwann zurückgehen und würde sagen: Okay, jetzt möchte ich aber auch mal negative Artikel sehen.
    Würde dann beispielsweise sagen: Okay, hier, da sehe ich da jetzt was Negatives.
    Dann würde ich halt hier draufklicken, um mir dann rauszusuchen, welchen negativen Artikel ich mir durchlesen kann, um halt auch mal die Gegenseite zu sehen.
    Wobei ich hier mal sagen muss, dass mir nicht sofort klar ist, welche Artikel jetzt hier in diese rote Kategorie einbezogen werden.
    Das fehlt mir eigentlich ein bisschen, weil dann muss ich mir wirklich die Artikel Bezeichnungen durchlesen, bevor ich überhaupt weiß, wie dieser Artikel bewertet ist.
    Und da hätte ich halt, würde ich mir jetzt, wenn ich als Nutzer auf diese Seite gehe, wünschen, dass ich da sofort sehe, ist jetzt der positive und welches der negative Artikel ist.
    \item \interview{P1} Und dadurch, dass ich mich jetzt schon ein bisschen mit der E-Mobilität auskenne, weiß ich, dass Rohstoffe für E-Auto Akku immer so ein kritisches Thema ist.
    Deswegen würde ich jetzt sagen okay, ich gehe jetzt da drauf, weil ich weiß, dass ich da dann wahrscheinlich was Negatives erwarten kann.
    Da sehe ich jetzt auch Staat, also genau diesen roten Punkt, den ich gesucht habe, der ist hier aufgeführt, also habe ich den gefunden.
    Wenn ich jetzt aber mich noch nicht damit auskennen würde, dann wäre mir das wahrscheinlich nicht so klar.
    Dann müsste ich mir wahrscheinlich alle Artikel durchlesen, um dann wirklich diese Gegenposition zu finden.
    Genau so würde ich mich jetzt weiter vorarbeiten.
    \item \interview{IV} Ja das wäre super, dass Du mir das noch kurz zeigst. 
    Sodass Du Dir noch ein paar weitere Artikel anschaust und dabei Deine Gedanken beschreibst.
    \item \interview{P1} Also das Erste, was mir dann jetzt noch auffallen würde, einfach, weil es jetzt gerade offen ist und weil es, bevor ich jetzt wieder nach einer anderen Kategorie suche, wäre jetzt \flqq E-Auto Förderung\frqq{}, weil ich da halt wieder Geld kriege.
    Wenn ich Geld kriege, nehme ich das natürlich immer mit.
    Genau.
    Und dann würde ich noch mal zurückgehen und würde auch noch mal in anderen Kategorien auch gucken.
    \item \interview{P1} Beispielsweise Industrie würde mich jetzt interessieren.
    Ja,\flqq Revolution oder Wegwerf-Auto?\frqq{}.
    Das ist natürlich so ein Titel, der direkt ins Auge springt.
    Deswegen würde ich mir den angucken und das ist jetzt sogar beides, also sowohl positiv als auch negativ betrachtet.
    Was natürlich auch immer schön ist, weil ich mir dadurch einen besseren Überblick verschaffen kann.
    \item {---------------------} [Aufgaben - FBA] {---------------------}
    \item \interview{IV} Finde und klicke den Artikel \flqq Harter Schlag für Hersteller Plugin Prämie fällt weg\frqq{}.
    \item \interview{P1} Okay, gut, jetzt ist die Frage, wenn ich hier draufklicke, sollte ich annehmen, sehe ich nur die, die hier unten sind.
    Ich war gerade am Überlegen, wie ich am besten alle aufgezählt bekomme, aber dann wahrscheinlich oben mit dem Knoten und dann sehe ich den hier direkt.
    \item \interview{IV} Und dann jetzt noch mal den Artikel \flqq Deutlich sauberer als gedacht\frqq{}.
    \item \interview{P1} Genau. Gleiches Vorgehen. Ich würde jetzt hier auf den ersten Knoten klicken, weil ich da alles aufgezählt bekomme.
    \item \interview{IV} Welcher Artikel ist am ähnlichsten zu \flqq Deutlich sauberer als gedacht?\frqq{}?
    \item \interview{P1} Also da würde ich jetzt erst mal auf die Sauberkeit achten.
    Also das ist ja so das Schlagwort dieses Artikels.
    Deswegen würde ich halt gucken, ob es da irgendwas gibt, was auch in Richtung Emissionen oder ähnliches geht.
    Gut, und in dem Fall sehe ich jetzt direkt, dass es hier ums Klima geht.
    Dadurch, dass ich ja jetzt schon in der Bubble drinnen bin, weiß ich okay, diese Sauberkeit hat was mit dem Klima zu tun.
    Deswegen würde ich da drauf gehen, auch wenn das jetzt eventuell vom Namen her eher das Gegenteil ist.
    Aber meiner Meinung nach hängt das natürlich trotzdem zusammen und würde mir halt sowohl auf positive als auch negative Art und Weise Informationen geben zu dem Thema. 
    Also der Artikel \flqq Von wegen nur das Klima retten\frqq{}.
    \item \interview{IV} Welcher Artikel unterscheidet sich am meisten von \flqq Deutlich sauberer als gedacht?\frqq{}?
    \item \interview{P1} Da würde ich jetzt so was sagen wie \flqq autonomes Fahren Erfolg oder Flop?\frqq{}, weil es halt einfach zumindest auf den ersten Blick erstmal nichts mit dem Klima zu tun hat.
    So über zehn Ecken gedacht natürlich schon, aber es ist halt die Frage.
    Da denkt man wahrscheinlich, wenn man sich vorher nie mit beschäftigt hat, auch nicht so drüber nach.
    Von daher würde ich jetzt persönlich das nehmen.
    \item {----------------------} [QUESI - FBA] {----------------------}
    \item {-------------------} [AttrakDiff2 - FBA] {-------------------}
    \item {-----------------------} [Fragebogen] {----------------------}
    \item \interview{IV} Welche Darstellungsformen für Online-News-Artikel sind Dir bekannt?
    \item \interview{P1} Eine Liste mit Thumbnails. 
    Aber nicht nur vertikal, sondern auch horizontal angeordnet.
    Ähnlich wie auf der YouTube-Startseite.
    Also ebenfalls mit Kategorien beziehungsweise Rubriken.
    Und dann halt typische Internetforen.
    \item \interview{IV} Wie häufig besuchst Du News-Webseiten?
    \item \interview{P1} Also wenn Du YouTube nicht dazuzählt, dann selten.+
    Wenn ich irgendwelchen News erfahre, dann meistens sogar über YouTube.
    \item \interview{IV} Inwieweit hast du bereits Vorerfahrung zum Thema E-Mobilität?
    \item \interview{P1} Ich fahre seit gut zwei Jahren oder so anderthalb Jahren eher selber ein Elektroauto.
    Also sagen wir mal so 2 bis 3 Jahre habe ich mich da schon mit beschäftigt. 
    \item \interview{IV} Und da auch hauptsächlich nur, um Deine Kaufentscheidung zu treffen oder wie genau?
    \item \interview{P1} Genau. 
    Also sowohl zur E-Mobilität selbst, aber auch insgesamt und natürlich auch zu meinem Tesla.
    \item \interview{IV} Könntest Du Dir vorstellen einer der beiden gezeigten Darstellungsformen persönlich zu nutzen?
    \item \interview{P1} Also ich denke, ich würde sogar eher die erste nutzen, weil hier - also kann natürlich sein, weil ich es einfach nicht gewohnt bin, aber hier habe ich das Gefühl, dass es umständlicher ist.
    Also es ist zwar, Du hast da diese Kategorisierung, die ich gut finde, also sowohl in Kategorien als auch in Bewertung.
    Das finde ich auf jeden Fall gut, aber ich find es halt nicht so übersichtlich, dass Du erst irgendwo draufklicken musst, um Dir die Artikel anzeigen zu lassen.
    Ich hätte lieber eine Übersicht, wo Du alle Artikel auf einmal sehen kannst.
    Die Liste bietet halt den Vorteil, dass Du von Anfang an die Schlagzeilen sehen und Dich da direkt entscheiden kannst, bevor Du Dir überhaupt Gedanken machen musst: In welche Kategorie will ich mich denn jetzt?
    \item \interview{IV} Hast Du bereits mit Graphen gearbeitet?
    \item \interview{P1} Ja, also das Typische in der Uni halt
    \item \interview{IV} Denkst Du, dass die Dir gezeigte neue Darstellungsform ein diverses Bild der Elektromobilität zeigt?
    \item \interview{P1} Also dadurch, dass Du hier diese Eingruppierung hast, finde ich schon, da es ja alle Aspekte beleuchtet und Dir auch zu einem gewissen Grad halt eine Information darüber gibt, ob ein Artikel jetzt positiv oder negativ ist.
    \item \interview{IV} Wie beeinflusst das Aussehen einer Webseite Deine Entscheidung, welche Informationen Du liest und wie lange Du auf der Seite bleibst?
    \item \interview{P1} Ich glaube schon sehr. 
    Also wenn ich jetzt beispielsweise noch mal das hier öffne.
    Ich meine, klar, es ist ja nur ein Prototyp. 
    Aber angenommen, ich hätte jetzt so eine Seite, wo ich keine Formatierungen oder sonst was habe, dann bestimmt stark.
    Wenn Du auf irgendeine Seite gehst und dann siehst Du so ein altes Design, dann denke ich mir halt eher okay, die Seite ist in die Jahre gekommen und dann bin ich da relativ schnell wieder weg.
    Also ja, da werde ich wahrscheinlich relativ schnell wieder von der Seite weg navigieren, weil ich halt eher aktuelle Seiten suche.
    \item {------------------} [Demografische Daten] {------------------}
    \item \interview{P1} 25, männlich, Bachelor, Informatiker, 
\end{itemize}}
\nolinenumbers
